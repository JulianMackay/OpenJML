\chapter{JML concepts}

\section{JML modifiers and Java annotations}

The Java Modeling Language was defined prior to the introduction of annotations in Java. Some, but not all, of the features of JML can now be textually represented as Java annotations. Currently JML supports both the old and new syntactic forms.

\subsubsection{Modifiers}
\label{Modifiers}

Modifiers are JML keywords that specify JML characteristics of methods, classes, fields, or variables. Examples are \texttt{pure}, \texttt{model}, and \texttt{ghost}. They are syntactically placed just like Java modifiers, such as \texttt{public}.

Each such modifier has an equivalent Java annotation. For example
\boxedexampleZ{ /*@ pure */ public int m(int i) { ... } }
can be written equivalently as 
\boxedexampleZ{ org.jmlspecs.annotation.Pure public int m(int i) { ... } }
The \texttt{org.jmlspecs.annotation} prefix can be made implicit in the usual way by including the import statement
\boxedexampleZ{import org.jmlspecs.annotation.Pure; }
Note that in the second form, the \texttt{pure} designation is now part of the \textit{Java} program and so the import of the
\texttt{org.jmlspecs.annotation} package must also be in the Java program, and the package must be available to the Java compiler.

All of the modifiers, their corresponding Java annotations, and the locations in which they may be used are described in \S\ref{modifiers}.

\subsubsection{Type modifiers}
Some modifiers are actually type modifiers. In particular \texttt{non\_null} and \texttt{nullable} are in this category. Thus the 
description of the previous subsection (\S\ref{Modifiers}) apply to these as well.

However, Java 1.8 allows Java annotations to be applied to types wherever type names may appear. For example 
\boxedexampleZ{ (@NonNull String)toUpper(s) } 
is allowed in Java 1.8 but is forbidden in Java 1.7. 

\textit{Need additional discussion of the change in JML for Java 1.8, especially for arrays.} %% TBD

\subsubsection{Method specification clauses}
\textit{This section will be added later.} %% TBD

\subsubsection{Class specification clauses}
\textit{This section will be added later.} %% TBD

\subsubsection{Statement specifications}

JML specifications that are statements within the body of a method have no equivalent as Java annotations. These include loop specifications, assert and assume statements, ghost declarations, set and debug statements, and specifications on individual statements.
 

\section{Model and Ghost}

\section{Visibility}

\section{JML types}

\section{Evaluation and well-formedness of JML expressions}

\section{Null and non-null references}

\section{Static and Instance}

\section{Location sets}

\section{Arithmetic modes}

\section{org.jmlspecs.lang package}
\label{JMLlang}

Some JML features are defined in the \texttt{org.jmlspecs.lang} package. The \texttt{org.jmlspecs.lang} package is included as a model import by default, just as the \texttt{java.lang} package is included by default in a Java file. \texttt{org.jmlspecs.lang.*}
contains these elements:
\begin{itemize}
\item \texttt{JML.informal(\textit{<string>}\texttt{)}} : This method is a replacement for (and is equivalent to) the informal expression syntax (\S\ref{JMLinformal}) \texttt{(* ... *)}. Both expressions return a boolean value, which is always \texttt{true}.

\item TBD

\end{itemize}

\chapter{Summary of JML Features}

The definition of the Java Modeling Language is contained in the JML reference manual.\cite{TBD}
This document does not repeat that definition in detail. However, the following sections summarize the features of JML, indicate what is and is not implemented in OpenJML, describes any extensions to JML contained in OpenJML, and includes
comments about relevant implementation aspects of OpenJML.

\section{JML Syntax}

\subsection{Syntax of JML specifications}

JML specifications may be written as Java annotations. Currently these are only implemented for modifiers (cf. section TBD). In Java 8, the use of Java annotations for JML features will be expanded.

JML specifications may also be written in specially formatted Java comments:
a JML specification includes everything between either (a) an opening {\tt /*\at} and closing {\tt */}
or (b) an opening {\tt //\at} and the next line ending character ({\tt \bs n}
or {\tt \bs r}) that is not within a string or character literal.

Such comments that occur within the body of a class or interface definition are
considered to be a specification of the class, a field, or a method, depending on the
kind of specification clause it is. JML specifications may also occur in the body of 
a method.

\paragraph{Obsolete syntax.} In previous versions of JML, JML specifications could be placed
within javadoc comments. Such specifications are no longer standard JML and are not supported by OpenJML.

\subsection{Conditional JML specifications}

JML has a mechanism for conditional specifications, based on a system of keys.
A key is an identifier (consisting of ASCII alphanumeric and underscore characters, and beginning with a non-digit).
A conditional JML comment is guarded by one or more positive or negative keys (or both).
The keys are placed just before the \at~character that is part of the opening sequence of the JML comment
(the {\tt //\at} or the {\tt /*\at}). Each key is preceded by a '+' or a '-' sign, to indicate whether it is a positive
or negative key, respectively. {\it No white-space is allowed}. If there is white-space anywhere between the
initial {\tt //} or {\tt /*} and the first \at~character, the comment will appear to be a normal Java comment and will be
silently ignored.

The keys are interpreted as follows. Each tool that processes the Java+JML input will have a means
(e.g. by command-line options) to specify the set of keys that are enabled.
\begin{itemize}
\item If the JML annotation has no keys, the annotation is always processed.
\item If there are only positive keys, the annotation is processed only if at least one of the keys is enabled.
\item If there are only negative keys, the annotation is processed unless one of the keys is enabled.
\item If there are both positive and negative keys, the annotation is processed only if (a) at least one of the
positive keys is enabled AND (b) none of the negative keys are enabled.
\end{itemize}

JML previously defined one conditional annotation: those that began with {\tt /*+\at} or {\tt //+\at}. ESC/Java2 also defined
{\tt /*-\at} and {\tt //-\at}. Both of these are now deprecated. OpenJML does have an option to enable the +-style comments.

The particular keys do not have any defined meaning in the JML reference manual. OpenJML implicitly enables the following keys:
\begin{itemize} 
\item {\bf ESC} : the \texttt{ESC} key is enabled when OpenJML is performing
ESC static checking; 
\item {\bf RAC} : the \texttt{RAC} key is enabled when OpenJML is performing Runtime-Assertion-Checking.
\item {\bf DEBUG} : The \texttt{DEBUG} key is not implicitly enabled. However it is defined as the key that enables the {\bf debug} JML statement. That is the {\bf debug} statement is ignored by default and is used by OpenJML if the user enables the DEBUG key.
\item {\bf OPENJML} : The OPENJML key is enabled whenever OpenJML is processing annotations (and presumably is not enabled by other tools).
\item {\bf KEY}: The \texttt{KEY} key is reserved for annotations recognized by the KeY tool~\cite{TBD}.  It is ignored by OpenJML.
\end{itemize}
Thus, for
example, one can turn off a non-executable assert statement for RAC-processing but retain it for ESC and for type-checking by writing //-RAC@ assert ... 


\subsection{Finding specification files and the refine statement}

JML allows specifications to be placed directly in the .java files that contain the implementation of methods and
classes. Indeed, specifications such as assert statements or loop invariants are necessarily placed directly in
a method body. Other specifications, such as class invariants and method pre- and post-conditions, may be placed in
auxiliary files. For classes which are only present as .class files and not as .java files, the auxiliary file is
a necessity.

Current JML allows one such auxiliary file per \texttt{.java} file or corresponding \texttt{.class} file. It is similar to the corresponding \texttt{.java} file except that
\begin{itemize} \nospace
\item it has a \texttt{.jml} suffix
\item it contains no method bodies (method declarations are terminated with semi-colons, as if they were abstract)
\item TBD - field initializations?
\end{itemize}
The \texttt{.jml} file must be in the same package as the corresponding \texttt{.java} file and has the same name, except for the suffix. It need not be in the same folder, though the tail of the path to the folder containing the \texttt{.jml} file must still correspond to the package containing the \texttt{.java} and \texttt{.jml} files.
If there is no source file, then there is a .jml file for each compilation unit that has a specification. All the nested, inner, or top-level classes that are defined in one Java compilation unit will have their specifications in one corresponding \texttt{.jml} file.

The search for specification files is analogous to the way in  which \texttt{.class} files are found on the {\it classpath}, except that the {\it specspath} is used instead. To find the specifications for a public top-level class {\it T}:
\begin{itemize} 
\item look in each element of the {\it specspath} (cf. section TBD), in order, for a fully-qualified file whose name is {\it T}{\tt .jml}.
If found, the contents of that file are used as the specifications of {\it T}.
\item if no such \texttt{.jml} file is found, look in each element of the {\it specspath}, in order, for a fully-qualified file whose name is {\it T}{\tt .java}.
\end{itemize}
There are two (silent) consequences of this search algorithm that can be confusing:
\begin{itemize}
\item If both a \texttt{.jml} and a \texttt{.java} file exist on the specspath and both contain JML specification text, the specifications in the \texttt{.java} file will be (silently) ignored.
\item If a \texttt{.java} file is listed on the command-line it will be compiled (for its Java content), but if it is not a member of an element of the {\it specspath}, it will (silently) not be used as the source of specifications for itself.
\end{itemize}

\paragraph{Obsolete syntax.} The {\tt refine} and {\tt refines} statements are no longer recognized.
The previous (complicated) method of finding specification files and merging the specifications
from multiple files is also no longer implemented. The only specification file suffix allowed is 
{\tt .jml}; the others --- {\tt .spec}, {\tt .refines-java}, {\tt .refines-spec}, {\tt .refines-jml} --- 
are no longer implemented.

In addition, the {\tt .jml} file is now sought before seeking the {\tt .java} file; if a {\tt .jml}
file is found anywhere in the specs path, then any specifications in the {\tt .java} file are 
ignored. This is a different search algorithm than was previously used.

\subsection{Model import statements}
\textit{This section will be added later.} %% TBD
Java \texttt{import} statements introduce class names into the namespace of a .java file.
JML has a \texttt{model import} statement:
\boxedexampleB{//@ model import ...}
The effect of a JML model import statement is the same as a Java import statement, except that the names imported by the JML statement are only visible within JML annotations. If the model import statement is within a .jml file, the imported names are
visible only within annotations in the .jml file, and not outside JML annotations and not in the .java file.

\textit{Note:} Most tools only approximately implement this feature. For example, see FIXME for a discussion of this feature in OpenJML.


\subsection{Modifiers}
\label{modifiers}

Modifiers are JML keywords that specify JML characteristics of methods, classes, fields, or variables. They are syntactically placed just like Java modifiers, such as \texttt{public}.

\begin{table}
\begin{tabular}{|l|l|c|c|c|c|c|}
\hline
JML Keyword & Java annotation & class & interface & method & field declaration & variable declaration \\
\hline
code & Code & & & X & & \\
code\_bigint\_math & CodeBigintMath & \\
code\_java\_math & CodeJavaMath & \\
code\_safe\_math & CodeSafeMath & \\
extract & Extract & \\
ghost & Ghost & & & & X & X \\
helper & Helper & & & X & & \\
instance & Instance & \\
model & Model & \\
monitored & Monitored & \\
non\_null & NonNull &  & & X & X & X \\
non\_null\_by\_default & NonNullByDefault & X & X & X & & \\
nullable & Nullable & & & X & X & X \\
nullable\_by\_default & NullableByDefault & X & X & X & &  \\
peer & Peer & \\
pure & Pure & X & X & X & & \\
query & Query & \\
readonly & Readonly & \\
rep & Rep & \\
secret & Secret & \\
spec\_bigint\_math & SpecBigintMath & \\
spec\_java\_math & SpecJavaMath & \\
spec\_protected & SpecProtected & \\
spec\_public & SpecPublic & \\
spec\_safe\_math & SpecSafeMath & \\
static & Static & \\
uninitialized & Uninitialized & \\

\hline
\end{tabular}
\caption{Summary of JML modifiers. All Java annotations are in the \texttt{org.jmlspecs.annotation} package.}
\label{Tab:modifiers}
\end{table}

\textit{CHeck the table; add section references; add where allowed; indicate which are type modifiers; turn headings 90 degrees.} %% TBD

\textit{Obsolete syntax}. JML no longer defines the modifier \texttt{weakly}. 

\subsection{Method specification clauses}
\textit{This section will be added later.} %% TBD

\subsubsection{Behaviors}

\subsection{Class specification clauses}
\textit{This section will be added later.} %% TBD

\subsection{Visibility of specifications}
\textit{This section will be added later.} %% TBD

\subsection{Statement specifications}
\textit{This section will be added later.} %% TBD

JML defines some statements that are used in the body of a method's implementation. These are not method specifications per se; 
rather, they are assertions or assumptions that are used to aid the proof of the specifications themselves, in the way that lemmas
are aids to proving a resulting theorem. They can also be used to state predicates that the user believes to be true, and wants checked, or assumptions that are true but are too difficult for the prover to prove itself.

\subsubsection{JML assert statement}
\textit{This section will be added later.} %% TBD

\subsubsection{Java assert statement}
\textit{This section will be added later.} %% TBD

\subsubsection{JML assume statement}
\textit{This section will be added later.} %% TBD

\subsubsection{Ghost declaration}
\textit{This section will be added later.} %% TBD

\subsubsection{set statement}
\textit{This section will be added later.} %% TBD


\subsubsection{debug statement}
\textit{This section will be added later.} %% TBD


\subsubsection{loop invariants}
\textit{This section will be added later.} %% TBD

\subsubsection{loop variants}
\textit{This section will be added later.} %% TBD

\subsubsection{loop assignable declarations}
\textit{This section will be added later.} %% TBD


\subsubsection{Refining statement specifications}
\textit{This section will be added later.} %% TBD

\subsection{JML expressions}

Expressions in JML annotations are Java expressions with three adjustments:
\begin{itemize}
\item Expressions with side-effects are not allowed. Specifically, JML excludes
\begin{itemize}[noitemsep,nolistsep]
\item the \texttt{++} and \texttt{--} pre- and post- increment and decrement operations
\item the assignment operator
\item assignment operators that combine an operation with assignment (e.g., \texttt{+=})
\item method invocations that are not explicitly declared \texttt{pure} (cf. \S TBD)
\end{itemize}
\item JML adds additional operators to the Java set of operators, discussed in subsection \S\ref{JMLoperators} below.
\item JML adds specific keywords that are used as constants or function-like expressions within JML expressions, discussed in subsection \S\ref{JMLkeywords} below
\end{itemize}

\subsubsection{JML operators}
\label{JMLoperators}

Table \ref{Tab:operators} lists all of the Java and JML operators, in order of precedence. The JML operators have identified by a comment and a reference to a subsection describing them. All of the JML keyword expressions in \S\ref{JMLkeywords} are primitives with precedence higher than any operator.

Most operators are left-associative, if associativity is applicable. The exceptions are TBD...

%% TBD - we really want a subsubsubsection instad of paragraph

\paragraph{JML implies (\texttt{==>}) and reverse implies (\texttt{<==})}
\label{JMLimplies}

The \texttt{==>} and \texttt{<==} are the JML implication and reverse implication operators. They are short circuit operations taking boolean operands with these equivalences:
\begin{itemize}[noitemsep,nolistsep]
\item \textit{p} \texttt{==>} \textit{q} is equivalent to \texttt{!}\textit{p}\texttt{ || }\textit{q}
\item \textit{p} \texttt{<==} \textit{q} is equivalent to \textit{p}\texttt{ || }\texttt{!}\textit{q}
\end{itemize}
Note that because of the short-circuit characteristic, \textit{p} \texttt{<==} \textit{q} is not quite equivalent to \textit{q} \texttt{==>} \textit{p}

\paragraph{JML equivalence (\texttt{<==>}) and inequivalence (\texttt{<=!=>})}
\label{JMLequivalence}

The \texttt{<==>} and \texttt{<=!=>} are the JML equivalence and inequivalence operators. They are equivalent to 
\texttt{==} and \texttt{!=} except that they take only boolean operands and have much lower precedence.

\paragraph{JML subtype operator (\texttt{<:})}
\label{JMLsubtype}
In JML expressions, \texttt{<:} denotes the subtype operation among classes and interfaces; the operands both have type \texttt{\bs TYPE} (cf. \S\ref{TBD}). Note that the subtype operation (despite not including the \texttt{=} character) includes both type equality and proper subtypes. Note also that in JML, types can express the full parameterized type, not just the erased type in runtime Java. More discussion of JML types is found in \S\ref{TBD}.

\paragraph{JML lock ordering operators (\texttt{<\#}) and \texttt{<\#=}})
\label{JMLlockorder}
The lock ordering oeprators are used to determine ordering among objects used for locking in a multi-threaded application; the operands are any Java objects. The only predefined property of these operators is that for any two 
object references \texttt{o} and \texttt{oo}, \texttt{o <\#= oo} is equivalent to \texttt{o == oo || o <\# oo}; that is \texttt{<\#} is like less than and \texttt{<\#=} is like less-than-or-equals. There is no predefined ordering among objects. The user must define an intended ordering with some axioms or invariants. An example of using the lock ordering operators for specification and reasoning about concurrency is found in \S\ref{TBD}.

\begin{table}
\begin{tabular}{|c|c|l|}
\hline
\texttt{new ()} \texttt{[] .} and method calls  \\ \hline
unary \texttt{+} unary \texttt{-} \texttt{~} \texttt{!} \texttt{(}typecast\texttt{)} & - & \\ \hline
\texttt{* / \%}  & L & \\ \hline
\texttt{+} (binary) \texttt{-} (binary) & L & \\ \hline
\texttt{<< >> >>>} & L & \\ \hline
\texttt{< <= > >= <: instanceof  <\# <\#=}  & - & \texttt{<:} is the JML subtype operation (\S\ref{JMLsubtype}); \\
						                                           & & <\# and <\#= are lock ordering operators (\S\ref{JMLlockorder}) \\ \hline
\texttt{== !=} & L & \\ \hline
\texttt{\&} & L & \\ \hline
\texttt{\^} & L &  \\ \hline
\texttt{|} & L &  \\ \hline
\texttt{\&\&} & L &  \\ \hline
\texttt{||} & L &  \\ \hline
\texttt{==>} \texttt{<==} & ? & JML implies and reverse implies (\S\ref{JMLimplies}) \\ \hline
\texttt{<==>} \texttt{<=!=>} & ? & JML equivalence and inequivalence (\S\ref{JMLequivalence}) \\ \hline
\texttt{?:} & - & \\ \hline
\texttt{= *= /= \%= += -= <<= >>= >>>= \&= \^= |=} & L & Java only\\ \hline

\end{tabular}
\label{Tab:precedence}
\caption{Java and JML operators, in order of precedence, from highest (most tightly binding) to lowest precedence. Operators on the same line have the same precedence. The associativity is given in the central column.}
\end{table}


TBD - add ++ -- into the table as Java only; check precedence

\subsubsection{JML expression keywords}
\label{JMLkeywords}
\textit{This section will be added later.} %% TBD

\paragraph{Referring to the result of a method: \texttt{\bs result}}

\paragraph{informal specification}. In some situations a specification needs to be expressed informally, in natural language, perhaps in anticipation of a formal expression or because a formal expression is too complex. The original JML informal expression is
{\center \texttt{(* ... *)}}
where the text between the delimiters is any natural language text not including the characters \texttt{*)}. 
An alternate expression (cf. \S\ref{JMLlang}) is the form \texttt{JML.informal(...)}, where the argument is a "`-delimited String, as permitted by Java.

In both cases the expression always has the value \texttt{true}.



\paragraph{Advanced quantifiers: \texttt{\bs max} \texttt{\bs min}
                \texttt{\bs num\_of} \texttt{\bs product} \texttt{\bs sum}}

\paragraph{Evaluating expressions in previous program states: \texttt{\bs old}, \texttt{\bs pre}, \texttt{\bs past}}

\paragraph{Newly allocated objects: \texttt{\bs fresh}}

\paragraph{Type expressions: \texttt{\bs type}, \texttt{\bs typeof}, \texttt{\bs elemtype}}

\paragraph{Loop expressions: \texttt{\bs index} and \texttt{\bs values}}


\paragraph{\texttt{\bs not\_modified}}

\paragraph{\texttt{\bs nonnullelements}}

\paragraph{\texttt{\bs not\_assigned}, \texttt{\bs only\_accessed}, \texttt{\bs only\_assigned}, \texttt{\bs only\_called}, \texttt{\bs only\_captured}}

TBD - reach, duration, space, working\_space

TBD - lockset, max

TBD - is\_initialized, invariant\_for

TBD - lblneg, lblpos, lbl (JML extension)

TBD - reach (object sets?)

\subsection{JML types}
Specifications are sometimes best written using infinite-precision mathematical types, rather than the fixed bit-width types of Java.
JML's arithmetic modes (\S\ref{TBD}) allow choosing among various numerical precisions.
In this section we simply note the type names that JML defines.

All of the Java type names are legal and useful in JML: \texttt{int short long byte char boolean double real} and class and interface types. In addition, JML defines the following:
\begin{itemize}
\item \texttt{\bs bigint} - the type of infinite-precision integers, represented as java.lang.BigInteger during run-time checking
\item \texttt{\bs real} - the type of mathematical real numbers, represented as TBD during runtime-checking
\item \texttt{\bs TYPE} - the type of JML type objects
\end{itemize}

The familiar operators are defined on values of the \texttt{\bs bigint} and \texttt{\bs real} types: unary and binary \texttt{+} and \texttt{-}, \texttt{*}, \texttt{/}, \texttt{\%}. Also, these types can be used in quantified expressions and variables of these types
can be declared as ghost or model variables.
  
The set of \texttt{\bs TYPE} values includes non-generic types such has \texttt{\bs type(org.lang.Object)}, fully parameterized generic types, such as \texttt{\bs type(org.utils.List<Integer>)}, and primitive types, such as \texttt{\bs type(int)}. 
The subtype operator (\texttt{<:}) is defined on values of type \texttt{\bs TYPE}.

TBD - what about other constructors or acccessors of TYPE values 


\subsection{Non-Null and Nullable}
\textit{This section will be added later.} %% TBD

\subsection{Observable purity: \texttt{\bs query} and \texttt{\bs secret}}
\textit{This section will be added later.} %% TBD

\subsection{Race condition detection}
\textit{This section will be added later.} %% TBD

\subsection{Arithmetic modes}
\textit{This section will be added later.} %% TBD

\subsection{Universe types}
\textit{This section will be added later.} %% TBD

\subsection{Dynamic frames}
\textit{This section will be added later.} %% TBD

\subsection{Code contracts}
\textit{This section will be added later.} %% TBD

\subsection{redundantly suffixes}
\textit{This section will be added later.} %% TBD

\subsection{nowarn lexical construct}

\textit{This section will be added later.} %% TBD


\section{Interaction with Java features}

\textit{This section will be added later.} %% TBD


\section{Other issues}

\subsection{Interaction with JSR-308}
\textit{This section will be added later.} %% TBD

\subsection{Interaction with FindBugs}

\textit{This section will be added later.} %% TBD

