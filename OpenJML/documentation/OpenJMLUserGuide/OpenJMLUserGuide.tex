%% Based on a TeXnicCenter-Template by Gyorgy SZEIDL.
%%%%%%%%%%%%%%%%%%%%%%%%%%%%%%%%%%%%%%%%%%%%%%%%%%%%%%%%%%%%%

%------------------------------------------------------------
%
\documentclass{report}%
%Options -- Point size:  10pt (default), 11pt, 12pt
%        -- Paper size:  letterpaper (default), a4paper, a5paper, b5paper
%                        legalpaper, executivepaper
%        -- Orientation  (portrait is the default)
%                        landscape
%        -- Print size:  oneside (default), twoside
%        -- Quality      final(default), draft
%        -- Title page   notitlepage, titlepage(default)
%        -- Columns      onecolumn(default), twocolumn
%        -- Equation numbering (equation numbers on the right is the default)
%                        leqno
%        -- Displayed equations (centered is the default)
%                        fleqn (equations start at the same distance from the right side)
%        -- Open bibliography style (closed is the default)
%                        openbib
% For instance the command
%           \documentclass[a4paper,12pt,leqno]{article}
% ensures that the paper size is a4, the fonts are typeset at the size 12p
% and the equation numbers are on the left side
%
\usepackage{amsmath}%
\usepackage{amsfonts}%
\usepackage{amssymb}%
\usepackage{graphicx}
\usepackage{caption}
\usepackage{color}
\definecolor{gray}{rgb}{0.8,0.8,0.8}
%% Language %%%%%%%%%%%%%%%%%%%%%%%%%%%%%%%%%%%%%%%%%%%%%%%%%
\usepackage[USenglish]{babel} %francais, polish, spanish, ...
\usepackage[T1]{fontenc}
\usepackage{textcomp}
\usepackage[ansinew]{inputenc}
\usepackage{makeidx}	  %% needed to create an index
\usepackage{hyperref}   %% sets hyperlinks within a pdf
\usepackage{rotating}
\usepackage{threeparttable}
\usepackage{verbatim}
\usepackage{tabularx}
\usepackage{enumitem}
\usepackage[parfill]{parskip} %% This avoicds indenting paragraphs and adds blank lines
%%\usepackage{listings}

%%\usepackage{lmodern} %Type1-font for non-english texts and characters


%% Packages for Graphics & Figures %%%%%%%%%%%%%%%%%%%%%%%%%%
\usepackage{graphicx} %%For loading graphic files
%\usepackage{subfig} %%Subfigures inside a figure
%\usepackage{tikz} %%Generate vector graphics from within LaTeX

\usepackage{mathptmx} %% Times fonts, for math as well
%% NOT WORKING?: \usepackage{mathpazo} %% Palatino fonts
%% NO: \usepackage{bookman} %% Bookman fonts for text, no change for math
%%\usepackage{newcent} %% NewCentury fonts for text, no change for math
%% Others: avant chancery charter helvet lucida palatino pslatex utopia times

%% DOES NOT SEEM TO WORK: \usepackage{fullpage}  %% Makes margins 1"

%% Please note:
%% Images can be included using \includegraphics{filename}
%% resp. using the dialog in the Insert menu.
%% 
%% The mode "LaTeX => PDF" allows the following formats:
%%   .jpg  .png  .pdf  .mps
%% 
%% The modes "LaTeX => DVI", "LaTeX => PS" und "LaTeX => PS => PDF"
%% allow the following formats:
%%   .eps  .ps  .bmp  .pict  .pntg


%% Commands for changing the Standardtexts %%%%%%%%%%%%%%%%%%
%\renewcommand{\ccname}{Copies}
%\renewcommand{\phonemsg}{Telephone}
%\renewcommand{\signmsgold}{Our signs}
%\renewcommand{\signmsgnew}{Our signs, our mail from}
%\renewcommand{\yourmailmsg}{Your signs, Your mail from}
%\renewcommand{\psname}{PS}
%\renewcommand{\enclname}{\tt Anlage(n)}
%\DeclareFontShape{OT1}{cmtt}{bx}{n}{<5><6><7><8><9><10><10.95><12><14.4><17.28><20.74><24.88>cmttb10}{}

\newcommand{\bs}{{\tt \char'134}}

\newcommand{\at}{{\tt \char'100}}
\newcommand{\boxspace}{\vspace*{6pt}}

\newlength{\savedrule}
\newlength{\thickrule}
\newlength{\framedwidth}
\newlength{\thickframedwidth}


\newcommand{\boxedexampleZ}[1]{\boxspace \\ \fbox{\begin{minipage}{\textwidth}{\tt #1}\end{minipage}}\boxspace\\*}
\newcommand{\boxedexampleS}[1]{\fbox{\begin{minipage}{\textwidth}{\tt #1}\end{minipage}}\boxspace\\*}

\newcommand{\boxedexample}[1]{\boxspace \\ \fbox{\begin{minipage}{5.50in}{\tt #1}\end{minipage}}\boxspace\\*}
%\newcommand{\boxedexampleA}[1]{\boxspace \\ \fbox{\parbox{6.00in}{\tt #1}}\boxspace\\*}

\newcommand{\boxedexampleA}[1]{\boxspace \\ \fbox{\begin{minipage}{6.00in}{\tt #1}\end{minipage}}\boxspace\\*}

\newcommand{\boxedexampleB}[1]{\boxspace \\ \fbox{\begin{minipage}{5.5in}{\tt #1}\end{minipage}}\boxspace\\*}

\newcommand{\boxedinput}[1]{\boxspace \\ \fbox{\begin{minipage}{6.00in}{\tt \verbatiminput{#1}}\end{minipage}}\boxspace\\*}

\newcommand{\boxedinputc}[2]{\boxspace \\ \fbox{\begin{minipage}{1.0\framedwidth}{\tt \verbatiminput{#1}}\end{minipage}} \setlength{\fboxrule}{\thickrule} \fbox{\begin{minipage}{1.0\thickframedwidth}{\tt \verbatiminput{#2}}\end{minipage}} \setlength{\fboxrule}{\savedrule} \boxspace\\*}

%\newcommand{\boxedinputc}[2]{\boxspace \\ \fbox{\begin{minipage}{1.0\framedwidth}{\tt \verbatiminput{#1}}\end{minipage}} \fcolorbox{black}{gray}{\begin{minipage}{1.0\framedwidth}{\tt \verbatiminput{#2}}\end{minipage}} \boxspace\\*}

%\newcommand{\boxedinputc}[2]{\boxspace \\ \fbox{\parbox{6.00in}{\tt \verbatiminput{#1}}} \fcolorbox{black}{gray}{\parbox{6.00in}{\tt \verbatiminput{#2}}} \boxspace\\*}

\newcommand{\source}{../../demos}

\newcommand{\demo}[1]{\boxedinputc{\source/#1.java}{\source/#1.txt}}

\newcommand{\nospace}{\setlength{\itemsep}{0in}\setlength{\parsep}{0in}\setlength{\parskip}{0in}\setlength{\topsep}{0in}}
\newcommand{\bae}[1]{\begin{array}{l}\\#1\\\end{array}}

\newcommand{\logand}{{\wedge}}
\newcommand{\logor}{{\vee}}
\newcommand{\imp}{{\Rightarrow}}
\newcommand{\inequiv}{{\neq}}
\newcommand{\smt}{{\tt \$\{SMTLIB\}}}
\newcommand{\cht}{\char'136}
\newcommand{\ind}{\hspace*{2em}}

\newcommand{\row}{}
%%\newcommand{\S}{section}  // TBD change this to the section symbol

%%\makeindex

%% The adjustment amounts depend on the font
%% -.5 for 12pt, -.875 for 10pt
%% FIXME: should these be set instead of adjusted?
\addtolength{\oddsidemargin}{-.5in}
\addtolength{\evensidemargin}{-.5in}
\addtolength{\textwidth}{1.0in}
\addtolength{\topmargin}{-.5in}
\addtolength{\textheight}{1.0in}

\setlength{\framedwidth}{6.00in}
\setlength{\thickframedwidth}{1.0\framedwidth}
\setlength{\savedrule}{1.0\fboxrule}
\setlength{\thickrule}{5.0\fboxrule}
\addtolength{\thickframedwidth}{-1.0\thickrule}
\addtolength{\thickframedwidth}{1.0\savedrule}

\begin{document}

\title{Does your software do what it should? \\
Specification and verification with the Java Modeling Language and OpenJML.
\\The OpenJML User Guide \\ \vspace{.2in} DRAFT IN PROGRESS}  

\author{David R. Cok \\ GrammaTech, Inc. }

\date{\today \\ \vspace*{.5in}
The most recent version of this document is available at \url{http://jmlspecs.sourceforge.net/OpenJMLUserGuide.pdf}.\\
 \vspace*{2.3in}
\parbox{\textwidth}{Copyright (c) 2010-2015 by David R. Cok. Permission is granted to make and distribute copies of this document for educational or research purposes, provided that the copyright notice and permission notice are preserved and acknowledgment is given in publications. 
Modified versions of the document may not be made. Please forward corrections to the author.
Incorporating this document within a larger collection, or distributing it for commercial purposes, or including it as part or all of a product for sale is allowed only by separate written permission from the author.}}

\maketitle
%
\tableofcontents

\chapter{Why specify? Why check?}

\chapter{Background of verification, JML, and OpenJML}

\chapter{Organization of this document}

This document addresses three related topics: how to read, write, and use specifications; the Java Modeling Language (JML) in which specifications are written; and the OpenJML tool that provides editing and checking support for Java programs using JML. 

These three topics are best learned in an interleaved fashion. The tutorial section (Part \ref{part:tutorial}) does just this. It introduces the simpler topics of specification, using Java programs with JML as the specification language, and using OpenJML as the tool to aid editing and checking, with motivating examples. A reader new to JML or to using specifications will find this tutorial to be the easiest introduction to the topics of this book. 

However, it is also useful to have a compact description of each of the JML language and the OpenJML tool. These descriptions are found in Parts \ref{part:jml} and \ref{part:openjml}) respectively. After some introduction, a reader may well want to 
take a break from the tutorial to read through and experiment with the details of JML and OpenJML. Once a reader has graduated from the tutorial and is specifying and verifying new examples, the description of JML serves as a summary of the JML language and the description of OpenJML is the user guide and reference manual for the tool. These two parts stand on there own.

Part \ref{part:contributing} contains information for those interested in contributing to the document. 
Contributions in the form of bug reports and experience reports with substantial use cases or experience in teaching are always welcome; this information can be shared directly with the developers or through the jmlspecs mailing list. 
Part \ref{part:contributing}, however, contains information primarily of interest to those developing and extending the OpenJML code itself.

The final part of the document, Part \ref{part:semantics}, describes details of how OpenJML translates the combination of Java and JML. This is not meant to be read through and is only intended for the reader interested in the detailed semantics of JML and the implementation of OpenJML.

FIXME - what about a mailing list

\chapter{Some details}

\section{Disambiguating `annotation'}. Formal specifications for code are often called annotations; in this document we often
use the term `JML annotations' to refer to specifications written in JML. There is also a specific syntactic construct in Java called 
`annotations': the interfaces labeled with `@' symbols that can modify various syntactic elements of Java. Thus the simple term
'annotation' can be ambiguous. The ambiguity is heightened by the fact that JML annotations, such as \texttt{/*@ pure */}, can 
be expressed as Java annotations, \texttt{@Pure}.

In this document, we will
generally disambiguate the term `annotation' as either 'JML annotation' or 'Java annotation'; if used alone, 'annotation' will generally
mean a JML annotation. 

\section{Syntactic conflicts with @}
For historical reasons, specifications are often written as structured programming language comments, with the \texttt{@} symbol denoting a comment containing specifications. Java comments begin with either \texttt{//} or \texttt{/*}; those comments that
contain JML specifications begin with \texttt{//@} or \texttt{/*@}. 
Similarly, \texttt{//} or \texttt{/*} are also used for comments in C and C++; the ANSI-C Specification Language also uses
\texttt{//@} or \texttt{/*@} to indicate comments containing specifications.

Unfortunately, since the \at symbol is also used for Java annotations, the following problem can arise. Some Java code is written something like (the particular Java annotation and its content are irrelevant)
\boxedexample{
@SuppressWarning("...") \\
class X
}
and then the user comments out the Java annotation without any whitespace:
\boxedexample{
//@SuppressWarning("...") \\
class X
}
Now JML tools will interpret the \texttt{//@} as the beginning of a JML annotation that will generally have parsing errors.

If the user includes whitespace, as in
\boxedexample{
// @SuppressWarning("...")
class X
}
there is then no problem. The workaround for this conflict is to edit the original Java source to include the whitespace. In some situations, placing all JML annotations in a .jml file may solve the problem; however, some tools, including OpenJML, may still parse
the .java file, including the erroneous apparently-JML annotations, even though those annotations are ignored when a .jml file is present.

\section{.jml files and .java files}

-TBD - .jml files hide annotations in .java files, except those in body of methods

\chapter{Quick start to OpenJML}

The details of installing and running OpenJML are presented in Part \ref{part:openjml}. However, an installation of the tool is needed to work through the tutorial. Some impatient readers may also wish to have a quick installation of the tool prior to diving into the full description. This section provides initial installation and use instructions.

\section{Installing OpenJML}

OpenJML is available as a command-line tool and as an Eclipse plug-in. Complete the following steps to install the command-line tool:
\begin{itemize}
\item Create or identify a directory (folder) in which to place the installation. Let \$OPENJML represent the path to this installation directory.
\item Download into \$OPENJML either the zip file at \url{http://jmlspecs.sourceforge.net/openjml.zip} or a gzipped tar file from \url{http://jmlspecs.sourceforge.net/openjml.tar.gz}
\item cd into the directory and either unzip (\texttt{unzip openjml.zip}) or untar (\texttt{tar xvzf openjml.tar.gz}) the downloaded file.
\item
\end{itemize}

\chapter{Other resources}

There are several other useful resources related to JML and OpenJML:
\begin{itemize}
\item \url{http://www.jmlspecs.org} is a web site describing current on JML, including references to many publications, other tools, and links to various groups using JML.
\item \url{http://www.jmlspecs.org/OldReleases/jmlrefman.pdf} is the official reference manual for JML, though it sometimes lags behind agreed-upon changes that are implemented in tools. (FIXME - make a better link)
\item \url{http://www.openjml.org} contains a set of on-line resources for OpenJML
\item \url{http://jmlspecs.sourceforge.net/OpenJMLUserGuide.pdf} is the most current version of this document
\item \url{http://jmlspecs.sourceforge.net/OpenJMLUserGuide.html} is an HTML version, with frames, of this document; \url{http://jmlspecs.sourceforge.net/OpenJMLUserGuide-onepage.html} is the same material in one large HTML page.
\item The source code for OpenJML, the original JML tools, and some other JML projects is contained in the jmlspecs sourceforge project at \url{http://sourceforge.net/projects/jmlspecs}.
\end{itemize}

There are also other tools that make use of JML. An incomplete list follows:
\begin{itemize}
\item Key - FIXME - need url
\item The previous generation of JML tools prior to OpenJML is available at \url{http://www.jmlspecs.org/download.shtml}.
\item FIXME - need others
\end{itemize}

\part{Tutorial introduction to specifying and checking Java programs}
\label{part:tutorial}

\input{00.tutorial.tex}

\part{The Java Modeling Language (JML)}
\label{part:jml}

\chapter{JML concepts}

\section{JML modifiers and Java annotations}

The Java Modeling Language was defined prior to the introduction of annotations in Java. Some, but not all, of the features of JML can now be textually represented as Java annotations. Currently JML supports both the old and new syntactic forms.

\subsubsection{Modifiers}
\label{Modifiers}

Modifiers are JML keywords that specify JML characteristics of methods, classes, fields, or variables. Examples are \texttt{pure}, \texttt{model}, and \texttt{ghost}. They are syntactically placed just like Java modifiers, such as \texttt{public}.

Each such modifier has an equivalent Java annotation. For example
\boxedexampleZ{ /*@ pure */ public int m(int i) { ... } }
can be written equivalently as 
\boxedexampleZ{ org.jmlspecs.annotation.Pure public int m(int i) { ... } }
The \texttt{org.jmlspecs.annotation} prefix can be made implicit in the usual way by including the import statement
\boxedexampleZ{import org.jmlspecs.annotation.Pure; }
Note that in the second form, the \texttt{pure} designation is now part of the \textit{Java} program and so the import of the
\texttt{org.jmlspecs.annotation} package must also be in the Java program, and the package must be available to the Java compiler.

All of the modifiers, their corresponding Java annotations, and the locations in which they may be used are described in \S\ref{modifiers}.

\subsubsection{Type modifiers}
Some modifiers are actually type modifiers. In particular \texttt{non\_null} and \texttt{nullable} are in this category. Thus the 
description of the previous subsection (\S\ref{Modifiers}) apply to these as well.

However, Java 1.8 allows Java annotations to be applied to types wherever type names may appear. For example 
\boxedexampleZ{ (@NonNull String)toUpper(s) } 
is allowed in Java 1.8 but is forbidden in Java 1.7. 

\textit{Need additional discussion of the change in JML for Java 1.8, especially for arrays.} %% TBD

\subsubsection{Method specification clauses}
\textit{This section will be added later.} %% TBD

\subsubsection{Class specification clauses}
\textit{This section will be added later.} %% TBD

\subsubsection{Statement specifications}

JML specifications that are statements within the body of a method have no equivalent as Java annotations. These include loop specifications, assert and assume statements, ghost declarations, set and debug statements, and specifications on individual statements.
 

\section{Model and Ghost}

\section{Visibility}

\section{JML types}

\section{Evaluation and well-formedness of JML expressions}

\section{Null and non-null references}

\section{Static and Instance}

\section{Location sets}

\section{Arithmetic modes}

\section{org.jmlspecs.lang package}
\label{JMLlang}

Some JML features are defined in the \texttt{org.jmlspecs.lang} package. The \texttt{org.jmlspecs.lang} package is included as a model import by default, just as the \texttt{java.lang} package is included by default in a Java file. \texttt{org.jmlspecs.lang.*}
contains these elements:
\begin{itemize}
\item \texttt{JML.informal(\textit{<string>}\texttt{)}} : This method is a replacement for (and is equivalent to) the informal expression syntax (\S\ref{JMLinformal}) \texttt{(* ... *)}. Both expressions return a boolean value, which is always \texttt{true}.

\item TBD

\end{itemize}

\chapter{Summary of JML Features}

The definition of the Java Modeling Language is contained in the JML reference manual.\cite{TBD}
This document does not repeat that definition in detail. However, the following sections summarize the features of JML, indicate what is and is not implemented in OpenJML, describes any extensions to JML contained in OpenJML, and includes
comments about relevant implementation aspects of OpenJML.

\section{JML Syntax}

\subsection{Syntax of JML specifications}

JML specifications may be written as Java annotations. Currently these are only implemented for modifiers (cf. section TBD). In Java 8, the use of Java annotations for JML features will be expanded.

JML specifications may also be written in specially formatted Java comments:
a JML specification includes everything between either (a) an opening {\tt /*\at} and closing {\tt */}
or (b) an opening {\tt //\at} and the next line ending character ({\tt \bs n}
or {\tt \bs r}) that is not within a string or character literal.

Such comments that occur within the body of a class or interface definition are
considered to be a specification of the class, a field, or a method, depending on the
kind of specification clause it is. JML specifications may also occur in the body of 
a method.

\paragraph{Obsolete syntax.} In previous versions of JML, JML specifications could be placed
within javadoc comments. Such specifications are no longer standard JML and are not supported by OpenJML.

\subsection{Conditional JML specifications}

JML has a mechanism for conditional specifications, based on a system of keys.
A key is an identifier (consisting of ASCII alphanumeric and underscore characters, and beginning with a non-digit).
A conditional JML comment is guarded by one or more positive or negative keys (or both).
The keys are placed just before the \at~character that is part of the opening sequence of the JML comment
(the {\tt //\at} or the {\tt /*\at}). Each key is preceded by a '+' or a '-' sign, to indicate whether it is a positive
or negative key, respectively. {\it No white-space is allowed}. If there is white-space anywhere between the
initial {\tt //} or {\tt /*} and the first \at~character, the comment will appear to be a normal Java comment and will be
silently ignored.

The keys are interpreted as follows. Each tool that processes the Java+JML input will have a means
(e.g. by command-line options) to specify the set of keys that are enabled.
\begin{itemize}
\item If the JML annotation has no keys, the annotation is always processed.
\item If there are only positive keys, the annotation is processed only if at least one of the keys is enabled.
\item If there are only negative keys, the annotation is processed unless one of the keys is enabled.
\item If there are both positive and negative keys, the annotation is processed only if (a) at least one of the
positive keys is enabled AND (b) none of the negative keys are enabled.
\end{itemize}

JML previously defined one conditional annotation: those that began with {\tt /*+\at} or {\tt //+\at}. ESC/Java2 also defined
{\tt /*-\at} and {\tt //-\at}. Both of these are now deprecated. OpenJML does have an option to enable the +-style comments.

The particular keys do not have any defined meaning in the JML reference manual. OpenJML implicitly enables the following keys:
\begin{itemize} 
\item {\bf ESC} : the \texttt{ESC} key is enabled when OpenJML is performing
ESC static checking; 
\item {\bf RAC} : the \texttt{RAC} key is enabled when OpenJML is performing Runtime-Assertion-Checking.
\item {\bf DEBUG} : The \texttt{DEBUG} key is not implicitly enabled. However it is defined as the key that enables the {\bf debug} JML statement. That is the {\bf debug} statement is ignored by default and is used by OpenJML if the user enables the DEBUG key.
\item {\bf OPENJML} : The OPENJML key is enabled whenever OpenJML is processing annotations (and presumably is not enabled by other tools).
\item {\bf KEY}: The \texttt{KEY} key is reserved for annotations recognized by the KeY tool~\cite{TBD}.  It is ignored by OpenJML.
\end{itemize}
Thus, for
example, one can turn off a non-executable assert statement for RAC-processing but retain it for ESC and for type-checking by writing //-RAC@ assert ... 


\subsection{Finding specification files and the refine statement}

JML allows specifications to be placed directly in the .java files that contain the implementation of methods and
classes. Indeed, specifications such as assert statements or loop invariants are necessarily placed directly in
a method body. Other specifications, such as class invariants and method pre- and post-conditions, may be placed in
auxiliary files. For classes which are only present as .class files and not as .java files, the auxiliary file is
a necessity.

Current JML allows one such auxiliary file per \texttt{.java} file or corresponding \texttt{.class} file. It is similar to the corresponding \texttt{.java} file except that
\begin{itemize} \nospace
\item it has a \texttt{.jml} suffix
\item it contains no method bodies (method declarations are terminated with semi-colons, as if they were abstract)
\item TBD - field initializations?
\end{itemize}
The \texttt{.jml} file must be in the same package as the corresponding \texttt{.java} file and has the same name, except for the suffix. It need not be in the same folder, though the tail of the path to the folder containing the \texttt{.jml} file must still correspond to the package containing the \texttt{.java} and \texttt{.jml} files.
If there is no source file, then there is a .jml file for each compilation unit that has a specification. All the nested, inner, or top-level classes that are defined in one Java compilation unit will have their specifications in one corresponding \texttt{.jml} file.

The search for specification files is analogous to the way in  which \texttt{.class} files are found on the {\it classpath}, except that the {\it specspath} is used instead. To find the specifications for a public top-level class {\it T}:
\begin{itemize} 
\item look in each element of the {\it specspath} (cf. section TBD), in order, for a fully-qualified file whose name is {\it T}{\tt .jml}.
If found, the contents of that file are used as the specifications of {\it T}.
\item if no such \texttt{.jml} file is found, look in each element of the {\it specspath}, in order, for a fully-qualified file whose name is {\it T}{\tt .java}.
\end{itemize}
There are two (silent) consequences of this search algorithm that can be confusing:
\begin{itemize}
\item If both a \texttt{.jml} and a \texttt{.java} file exist on the specspath and both contain JML specification text, the specifications in the \texttt{.java} file will be (silently) ignored.
\item If a \texttt{.java} file is listed on the command-line it will be compiled (for its Java content), but if it is not a member of an element of the {\it specspath}, it will (silently) not be used as the source of specifications for itself.
\end{itemize}

\paragraph{Obsolete syntax.} The {\tt refine} and {\tt refines} statements are no longer recognized.
The previous (complicated) method of finding specification files and merging the specifications
from multiple files is also no longer implemented. The only specification file suffix allowed is 
{\tt .jml}; the others --- {\tt .spec}, {\tt .refines-java}, {\tt .refines-spec}, {\tt .refines-jml} --- 
are no longer implemented.

In addition, the {\tt .jml} file is now sought before seeking the {\tt .java} file; if a {\tt .jml}
file is found anywhere in the specs path, then any specifications in the {\tt .java} file are 
ignored. This is a different search algorithm than was previously used.

\subsection{Model import statements}
\textit{This section will be added later.} %% TBD
Java \texttt{import} statements introduce class names into the namespace of a .java file.
JML has a \texttt{model import} statement:
\boxedexampleB{//@ model import ...}
The effect of a JML model import statement is the same as a Java import statement, except that the names imported by the JML statement are only visible within JML annotations. If the model import statement is within a .jml file, the imported names are
visible only within annotations in the .jml file, and not outside JML annotations and not in the .java file.

\textit{Note:} Most tools only approximately implement this feature. For example, see FIXME for a discussion of this feature in OpenJML.


\subsection{Modifiers}
\label{modifiers}

Modifiers are JML keywords that specify JML characteristics of methods, classes, fields, or variables. They are syntactically placed just like Java modifiers, such as \texttt{public}.

\begin{table}
\begin{tabular}{|l|l|c|c|c|c|c|}
\hline
JML Keyword & Java annotation & class & interface & method & field declaration & variable declaration \\
\hline
code & Code & & & X & & \\
code\_bigint\_math & CodeBigintMath & \\
code\_java\_math & CodeJavaMath & \\
code\_safe\_math & CodeSafeMath & \\
extract & Extract & \\
ghost & Ghost & & & & X & X \\
helper & Helper & & & X & & \\
instance & Instance & \\
model & Model & \\
monitored & Monitored & \\
non\_null & NonNull &  & & X & X & X \\
non\_null\_by\_default & NonNullByDefault & X & X & X & & \\
nullable & Nullable & & & X & X & X \\
nullable\_by\_default & NullableByDefault & X & X & X & &  \\
peer & Peer & \\
pure & Pure & X & X & X & & \\
query & Query & \\
readonly & Readonly & \\
rep & Rep & \\
secret & Secret & \\
spec\_bigint\_math & SpecBigintMath & \\
spec\_java\_math & SpecJavaMath & \\
spec\_protected & SpecProtected & \\
spec\_public & SpecPublic & \\
spec\_safe\_math & SpecSafeMath & \\
static & Static & \\
uninitialized & Uninitialized & \\

\hline
\end{tabular}
\caption{Summary of JML modifiers. All Java annotations are in the \texttt{org.jmlspecs.annotation} package.}
\label{Tab:modifiers}
\end{table}

\textit{CHeck the table; add section references; add where allowed; indicate which are type modifiers; turn headings 90 degrees.} %% TBD

\textit{Obsolete syntax}. JML no longer defines the modifier \texttt{weakly}. 

\subsection{Method specification clauses}
\textit{This section will be added later.} %% TBD

\subsubsection{Behaviors}

\subsection{Class specification clauses}
\textit{This section will be added later.} %% TBD

\subsection{Visibility of specifications}
\textit{This section will be added later.} %% TBD

\subsection{Statement specifications}
\textit{This section will be added later.} %% TBD

JML defines some statements that are used in the body of a method's implementation. These are not method specifications per se; 
rather, they are assertions or assumptions that are used to aid the proof of the specifications themselves, in the way that lemmas
are aids to proving a resulting theorem. They can also be used to state predicates that the user believes to be true, and wants checked, or assumptions that are true but are too difficult for the prover to prove itself.

\subsubsection{JML assert statement}
\textit{This section will be added later.} %% TBD

\subsubsection{Java assert statement}
\textit{This section will be added later.} %% TBD

\subsubsection{JML assume statement}
\textit{This section will be added later.} %% TBD

\subsubsection{Ghost declaration}
\textit{This section will be added later.} %% TBD

\subsubsection{set statement}
\textit{This section will be added later.} %% TBD


\subsubsection{debug statement}
\textit{This section will be added later.} %% TBD


\subsubsection{loop invariants}
\textit{This section will be added later.} %% TBD

\subsubsection{loop variants}
\textit{This section will be added later.} %% TBD

\subsubsection{loop assignable declarations}
\textit{This section will be added later.} %% TBD


\subsubsection{Refining statement specifications}
\textit{This section will be added later.} %% TBD

\subsection{JML expressions}

Expressions in JML annotations are Java expressions with three adjustments:
\begin{itemize}
\item Expressions with side-effects are not allowed. Specifically, JML excludes
\begin{itemize}[noitemsep,nolistsep]
\item the \texttt{++} and \texttt{--} pre- and post- increment and decrement operations
\item the assignment operator
\item assignment operators that combine an operation with assignment (e.g., \texttt{+=})
\item method invocations that are not explicitly declared \texttt{pure} (cf. \S TBD)
\end{itemize}
\item JML adds additional operators to the Java set of operators, discussed in subsection \S\ref{JMLoperators} below.
\item JML adds specific keywords that are used as constants or function-like expressions within JML expressions, discussed in subsection \S\ref{JMLkeywords} below
\end{itemize}

\subsubsection{JML operators}
\label{JMLoperators}

Table \ref{Tab:operators} lists all of the Java and JML operators, in order of precedence. The JML operators have identified by a comment and a reference to a subsection describing them. All of the JML keyword expressions in \S\ref{JMLkeywords} are primitives with precedence higher than any operator.

Most operators are left-associative, if associativity is applicable. The exceptions are TBD...

%% TBD - we really want a subsubsubsection instad of paragraph

\paragraph{JML implies (\texttt{==>}) and reverse implies (\texttt{<==})}
\label{JMLimplies}

The \texttt{==>} and \texttt{<==} are the JML implication and reverse implication operators. They are short circuit operations taking boolean operands with these equivalences:
\begin{itemize}[noitemsep,nolistsep]
\item \textit{p} \texttt{==>} \textit{q} is equivalent to \texttt{!}\textit{p}\texttt{ || }\textit{q}
\item \textit{p} \texttt{<==} \textit{q} is equivalent to \textit{p}\texttt{ || }\texttt{!}\textit{q}
\end{itemize}
Note that because of the short-circuit characteristic, \textit{p} \texttt{<==} \textit{q} is not quite equivalent to \textit{q} \texttt{==>} \textit{p}

\paragraph{JML equivalence (\texttt{<==>}) and inequivalence (\texttt{<=!=>})}
\label{JMLequivalence}

The \texttt{<==>} and \texttt{<=!=>} are the JML equivalence and inequivalence operators. They are equivalent to 
\texttt{==} and \texttt{!=} except that they take only boolean operands and have much lower precedence.

\paragraph{JML subtype operator (\texttt{<:})}
\label{JMLsubtype}
In JML expressions, \texttt{<:} denotes the subtype operation among classes and interfaces; the operands both have type \texttt{\bs TYPE} (cf. \S\ref{TBD}). Note that the subtype operation (despite not including the \texttt{=} character) includes both type equality and proper subtypes. Note also that in JML, types can express the full parameterized type, not just the erased type in runtime Java. More discussion of JML types is found in \S\ref{TBD}.

\paragraph{JML lock ordering operators (\texttt{<\#}) and \texttt{<\#=}})
\label{JMLlockorder}
The lock ordering oeprators are used to determine ordering among objects used for locking in a multi-threaded application; the operands are any Java objects. The only predefined property of these operators is that for any two 
object references \texttt{o} and \texttt{oo}, \texttt{o <\#= oo} is equivalent to \texttt{o == oo || o <\# oo}; that is \texttt{<\#} is like less than and \texttt{<\#=} is like less-than-or-equals. There is no predefined ordering among objects. The user must define an intended ordering with some axioms or invariants. An example of using the lock ordering operators for specification and reasoning about concurrency is found in \S\ref{TBD}.

\begin{table}
\begin{tabular}{|c|c|l|}
\hline
\texttt{new ()} \texttt{[] .} and method calls  \\ \hline
unary \texttt{+} unary \texttt{-} \texttt{~} \texttt{!} \texttt{(}typecast\texttt{)} & - & \\ \hline
\texttt{* / \%}  & L & \\ \hline
\texttt{+} (binary) \texttt{-} (binary) & L & \\ \hline
\texttt{<< >> >>>} & L & \\ \hline
\texttt{< <= > >= <: instanceof  <\# <\#=}  & - & \texttt{<:} is the JML subtype operation (\S\ref{JMLsubtype}); \\
						                                           & & <\# and <\#= are lock ordering operators (\S\ref{JMLlockorder}) \\ \hline
\texttt{== !=} & L & \\ \hline
\texttt{\&} & L & \\ \hline
\texttt{\^} & L &  \\ \hline
\texttt{|} & L &  \\ \hline
\texttt{\&\&} & L &  \\ \hline
\texttt{||} & L &  \\ \hline
\texttt{==>} \texttt{<==} & ? & JML implies and reverse implies (\S\ref{JMLimplies}) \\ \hline
\texttt{<==>} \texttt{<=!=>} & ? & JML equivalence and inequivalence (\S\ref{JMLequivalence}) \\ \hline
\texttt{?:} & - & \\ \hline
\texttt{= *= /= \%= += -= <<= >>= >>>= \&= \^= |=} & L & Java only\\ \hline

\end{tabular}
\label{Tab:precedence}
\caption{Java and JML operators, in order of precedence, from highest (most tightly binding) to lowest precedence. Operators on the same line have the same precedence. The associativity is given in the central column.}
\end{table}


TBD - add ++ -- into the table as Java only; check precedence

\subsubsection{JML expression keywords}
\label{JMLkeywords}
\textit{This section will be added later.} %% TBD

\paragraph{Referring to the result of a method: \texttt{\bs result}}

\paragraph{informal specification}. In some situations a specification needs to be expressed informally, in natural language, perhaps in anticipation of a formal expression or because a formal expression is too complex. The original JML informal expression is
{\center \texttt{(* ... *)}}
where the text between the delimiters is any natural language text not including the characters \texttt{*)}. 
An alternate expression (cf. \S\ref{JMLlang}) is the form \texttt{JML.informal(...)}, where the argument is a "`-delimited String, as permitted by Java.

In both cases the expression always has the value \texttt{true}.



\paragraph{Advanced quantifiers: \texttt{\bs max} \texttt{\bs min}
                \texttt{\bs num\_of} \texttt{\bs product} \texttt{\bs sum}}

\paragraph{Evaluating expressions in previous program states: \texttt{\bs old}, \texttt{\bs pre}, \texttt{\bs past}}

\paragraph{Newly allocated objects: \texttt{\bs fresh}}

\paragraph{Type expressions: \texttt{\bs type}, \texttt{\bs typeof}, \texttt{\bs elemtype}}

\paragraph{Loop expressions: \texttt{\bs index} and \texttt{\bs values}}


\paragraph{\texttt{\bs not\_modified}}

\paragraph{\texttt{\bs nonnullelements}}

\paragraph{\texttt{\bs not\_assigned}, \texttt{\bs only\_accessed}, \texttt{\bs only\_assigned}, \texttt{\bs only\_called}, \texttt{\bs only\_captured}}

TBD - reach, duration, space, working\_space

TBD - lockset, max

TBD - is\_initialized, invariant\_for

TBD - lblneg, lblpos, lbl (JML extension)

TBD - reach (object sets?)

\subsection{JML types}
Specifications are sometimes best written using infinite-precision mathematical types, rather than the fixed bit-width types of Java.
JML's arithmetic modes (\S\ref{TBD}) allow choosing among various numerical precisions.
In this section we simply note the type names that JML defines.

All of the Java type names are legal and useful in JML: \texttt{int short long byte char boolean double real} and class and interface types. In addition, JML defines the following:
\begin{itemize}
\item \texttt{\bs bigint} - the type of infinite-precision integers, represented as java.lang.BigInteger during run-time checking
\item \texttt{\bs real} - the type of mathematical real numbers, represented as TBD during runtime-checking
\item \texttt{\bs TYPE} - the type of JML type objects
\end{itemize}

The familiar operators are defined on values of the \texttt{\bs bigint} and \texttt{\bs real} types: unary and binary \texttt{+} and \texttt{-}, \texttt{*}, \texttt{/}, \texttt{\%}. Also, these types can be used in quantified expressions and variables of these types
can be declared as ghost or model variables.
  
The set of \texttt{\bs TYPE} values includes non-generic types such has \texttt{\bs type(org.lang.Object)}, fully parameterized generic types, such as \texttt{\bs type(org.utils.List<Integer>)}, and primitive types, such as \texttt{\bs type(int)}. 
The subtype operator (\texttt{<:}) is defined on values of type \texttt{\bs TYPE}.

TBD - what about other constructors or acccessors of TYPE values 


\subsection{Non-Null and Nullable}
\textit{This section will be added later.} %% TBD

\subsection{Observable purity: \texttt{\bs query} and \texttt{\bs secret}}
\textit{This section will be added later.} %% TBD

\subsection{Race condition detection}
\textit{This section will be added later.} %% TBD

\subsection{Arithmetic modes}
\textit{This section will be added later.} %% TBD

\subsection{Universe types}
\textit{This section will be added later.} %% TBD

\subsection{Dynamic frames}
\textit{This section will be added later.} %% TBD

\subsection{Code contracts}
\textit{This section will be added later.} %% TBD

\subsection{redundantly suffixes}
\textit{This section will be added later.} %% TBD

\subsection{nowarn lexical construct}

\textit{This section will be added later.} %% TBD


\section{Interaction with Java features}

\textit{This section will be added later.} %% TBD


\section{Other issues}

\subsection{Interaction with JSR-308}
\textit{This section will be added later.} %% TBD

\subsection{Interaction with FindBugs}

\textit{This section will be added later.} %% TBD



\part{The OpenJML tool}
\label{part:openjml}

\chapter{Introduction}

\section{OpenJML}

OpenJML is a tool for processing Java Modeling Language (JML) specifications of Java programs. 
The tool parses and type-checks the specifications and
performs static or run-time checking of the validity of the specifications. 
Other tools are anticipated, such as test case generation and
 an enhanced version of javadoc that includes the specifications in the
javadoc documentation.

The functionality is available 
\begin{itemize}
\item as a command-line tool to do type-checking, static checking or runtime checking,
\item as an Eclipse plug-in to perform those tasks, and
\item programmatically from a user's Java program
\end{itemize}

OpenJML uses program specifications written in JML, the Java Modeling Language, and
it is constructed by extending OpenJDK, the open source Java compiler.
OpenJML currently requires Java 1.7 to run.
Releases of Java can be obtained from the Oracle release site (\url{http://www.oracle.com/technetwork/java/javase/downloads}).

The source code for OpenJML is kept in SourceForge as a module of the 
JML Project (\url{https://sourceforge.net/p/jmlspecs/code/HEAD/tree/}).
The JMLAnnotations and Specs projects (also modules of JML) are used by OpenJML.
Information about creating a developer's environment for the OpenJML source
can be found below (section \ref{Development}).
 
\subsection{Command-line tool}

The OpenJML command line tool can be downloaded from
\url{http://jmlspecs.sourceforge.net/openjml.tar.gz}.

The command line tool is described in chapter \ref{CommandLineTool}.

\subsection{Eclipse plug-in}

The Update site for the Eclipse plug-in that encapsulates the OpenJML tool
is \url{http://jmlspecs.sourceforge.net/openjml-updatesite}.

The plug-in is described in section \ref{EclipsePlugin} and in the online documentation available in Eclipse Help.


\subsection{Development of OpenJML}
\label{Development}

Developers wishing to contribute to OpenJML can retrieve a project-set file to download source code from SVN and create the corresponding projects within Eclipse from 
\url{http://jmlspecs.sourceforge.net/OpenJML-projectSet.psf}.
<p>Alternately, the set of SVN commands needed to checkout all the pieces of the
OpenJML source code into the directory structure expected by Eclipse is found at this link: 
\url{http://jmlspecs.sourceforge.net/svn_commands}.

The general instructions for setting up a development environment are found at the JML wiki: \url{https://sourceforge.net/apps/trac/jmlspecs/wiki/OpenJmlSetup}.



\section{JML}

The Java Modeling Language (JML) is a language that enables logical assertions
to be made about Java programs. The assertions are expressed as structured 
Java comments or Java annotations. Various tools can then read the JML 
information and do static checking, runtime checking, display for documentation,
or other useful tasks.

More information about JML can be found on the JML web site: 
\url{http://www.jmlspecs.org}.
The information includes publications, a list of groups using or contributing to JML, 
mailing lists, etc.
There is also a SourceForge project for JML : \url{https://sourceforge.net/projects/jmlspecs/}.

\section{OpenJDK}

OpenJDK (\url{http://openjdk.net}) is the project that produces the Java JDK and JRE releases.
OpenJML extends OpenJDK to produce the OpenJML tools. OpenJML is a 
fully encapsulated, stand-alone tool, so the OpenJDK foundation is only of interest to OpenJML developers.
Users, however, can be assured that OpenJML is built on 'official' Java tooling and can readily stay up 
to date with changes in the Java language.

\section{License}

The OpenJML command-line tool is built from OpenJDK, which is licensed under GPLv.2 (\url{http://openjdk.java.net/legal/}).
Hence OpenJML is correspondingly licensed.

The OpenJML plug-in is a pure Eclipse plug-in, and therefore is not required to be licensed under the EPL.


The source code for both tools is available as a SourceForge project at
\url{https://sourceforge.net/p/jmlspecs/code/HEAD/tree/OpenJML/trunk}.


\chapter{The command-line tool}
\label{CommandLineTool}
\section{Installation and System Requirements}

The command-line tool is supplied as a .zip or a .tar.gz file, downloadable from \url{http://jmlspecs.sourceforge.net/}.
Download the file to a directory of your choice and unzip or untar it in place.
It contains the following files:
\begin{itemize}\nospace
\item openjml.jar - the main jar file for the application
\item jmlruntime.jar - a library needed on the classpath when running OpenJML's runtime-assertion-checking
\item jmlspecs.jar - a library containing specification files
\item openjml-template.properties - a sample file, which should be copied and renamed {\tt openjml.properties}, containing definitions of properties whose values depend on your local system
\item LICENSE.rtf - a copy of the modified GPL license that applies to OpenJDK and OpenJML
\item OpenJMLUserGuide.pdf - this document
\end{itemize}

You must run OpenJML in a Java 1.7 JRE; it is not compatible with Java 1.8 or 1.9.

You should ensure that the {\tt jmlruntime.jar} and {\tt jmlspecs.jar} files remain in the same folder as the {\tt openjml.jar} file.

In addition to OpenJML itself, you will also need a SMT solver if you intend to use the static checking capability of OpenJML.

TBD - more on installing SMT solvers

\section{Running OpenJML}
\label{Running}

To run OpenJML, be sure that the \texttt{java} command uses a 1.7 JVM and use the following command line. Here \textit{\$OPENJML} designates the folder in which the {\tt openjml.jar} file resides.
\boxedexampleZ{
java -jar \textit{\$OPENJML}/openjml.jar {\it <options>} {\it <files>}
}

Here {\it <files>} and {\it <options>} stand for text described below.

The following command is currently a viable alternative as well.
\boxedexampleZ{
java -cp \textit{\$OPENJML}/openjml.jar org.jmlspecs.openjml.Main {\it <options>} {\it <files>}
}

The valid options are listed in Table \ref{Tab:Options} and are described in subsections below. Options and files can appear in any order.

\subsection{Exit values}

When OpenJML runs as a command-line tool, it emits one of several values on exit:
\begin{itemize}[noitemsep,nolistsep]
\item 0 (\texttt{EXIT\_OK}) : successful operation, no errors, there may be warnings (including static checking warnings)
\item 1 (\texttt{EXIT\_ERROR}) : normal operation, but with errors (parsing or type-checking)
\item 2 (\texttt{EXIT\_CMDERR}) : an error in the formulation of the command-line, such as invalid options
\item 3 (\texttt{EXIT\_SYSERR}) : a system error, such as out of memory
\item 4 (\texttt{EXIT\_ABNORMAL}) : a fatal error, such as a program crash or internal inconsistency, caused by an internal bug
\end{itemize}

In the itemized list above, the symbolic names are defined in {\tt org.jmlspecs.openjml.Main}. For example, when executing OpenJML programmatically (cf. section TBD), the user's Java program should use the 
symbol {\tt org.jmlspecsopenjml.Main.EXIT\_ERROR}, rather than a specific integer.

Compiler warnings and static checking warnings will be reported as errors if the \texttt{-Werror} option is used. This may change an \texttt{EXIT\_OK} result to an \texttt{EXIT\_ERROR} result.



\subsection{Files}

In the command templates above, {\it <files>} refers to a list of {\tt .java} or {\tt .jml} files.
Each file must be specified with an absolute file system path or with a path relative
to the current working directory (in particular, not with respect to the classpath or
the sourcepath). 

You can also specify directories on the command line using the {\tt -dir} and {\tt -dirs} options.
The {\tt -dir {\it <directory>}} option indicates that the {\it <directory>} value (an absolute or
relative path to a folder) should be understood as a folder; all {\tt .java} or specification files
recursively (TBD - check recursively) within the folder are included as if they were individually listed on the command-line. The
{\tt -dirs} option indicates that each one of the remaining command-line arguments is interpreted as 
either a source file (if it is a file with a {\tt .java} or {\tt .jml}suffix) or as a folder (if it is a folder)
whose contents are processed as if listed on the command-line. Note that the {\tt -dirs} option must be
the last option.

As described later in section \ref{SpecFiles}, JML specifications for Java programs can be placed either in the 
{\tt .java} files themselves or in auxiliary {\tt .jml} files. The format
of {\tt .jml} files is defined by JML. OpenJML can type-check 
{\tt .jml} files as well as {\tt .java} files if they are placed on the 
command-line. Doing so can be useful to check the syntax in a specific
{\tt .jml} file, but is usually not necessary: when a {\tt .java} file is
processed by OpenJML, the corresponding {\tt .jml} file is automatically found (cf. \ref{TBD}).

\begin{table} \small
%%\centering
\parbox{.5\textwidth}{
\begin{tabular}{|l|p{1.4in}|}
\hline
\multicolumn{2}{|c|}{Options specific to JML} \\
\hline
-- & no more options \\ \hline
-check & [\ref{OptionsTools}] typecheck only ({\tt -command check})\\ \hline
-checkSpecsPath & [\ref{OptionsJML}] warn about non-existent specs path entries\\ \hline
-command {\it <action>}& [\ref{OptionsTools}] which action to do: check esc rac compile\\ \hline
-compile & [\ref{OptionsTools}] TBD\\ \hline
-counterexample & [\ref{OptionsESC}] show a counterexample for failed static checks\\ \hline
-dir {\it <dir>} & [\ref{OptionsJML}] argument is a folder or file \\ \hline
-dirs & [\ref{OptionsJML}] remaining arguments are folders or files\\ \hline
-esc & [\ref{OptionsTools}] do static checking ({\tt -command esc}) \\ \hline
-internalRuntime & [\ref{OptionsJML}] add internal runtime library to classpath \\ \hline
-internalSpecs & [\ref{OptionsJML}] add internal specs library to specspath \\ \hline
-java & [\ref{OptionsTools}] use the native OpenJDK tool\\ \hline
-jml & [\ref{OptionsTools}] process JML constructs \\ \hline
-jmldebug & [\ref{OptionsDebugging}] very verbose output (includes -progress) \\ \hline
-jmlverbose & [\ref{OptionsDebugging}] JML-specific verbose output\\ \hline
-keys & [\ref{OptionsJML}] define keys for optional annotations \\ \hline
-method & \\ \hline
-nonnullByDefault & [\ref{OptionsJML}] values are not null by default \\ \hline
-normal & [\ref{OptionsDebugging}] \\ \hline
-nullableByDefault & [\ref{OptionsJML}] values may be null by default\\ \hline
-progress & [\ref{OptionsDebugging}] \\ \hline
-purityCheck & [\ref{OptionsJML}] check for purity \\ \hline
-quiet & [\ref{OptionsDebugging}] no informational output \\ \hline
-rac & [\ref{OptionsTools}] compile runtime assertion checks ({\tt -command rac})\\ \hline
-racCheckAssumptions & [\ref{OptionsRAC}] enables (default on) checking assume statements as if they were asserts \\ \hline
-racCompileToJavaAssert & [\ref{OptionsRAC}] compile RAC checks using Java asserts \\ \hline
-racJavaChecks & [\ref{OptionsRAC}] enables (default on) performing JML checking of violated Java features \\ \hline
\end{tabular}
}
\qquad
%%\centering
\parbox{.5\textwidth}{

\begin{tabular}{|l|p{1.4in}|}
\hline
\multicolumn{2}{|c|}{JML options, continued} \\
\hline
-racShowSource & [\ref{OptionsRAC}] includes source location in RAC warning messages \\ \hline
-showNotImplemented & warn if feature not implemented\\ \hline
-specspath & [\ref{OptionsJML}] location of specs files\\ \hline
-stopIfParseErrors & stop if there are any parse errors \\ \hline
-subexpressions & [\ref{OptionsESC}] show subexpression detail for failed static checks\\ \hline
-trace & [\ref{OptionsESC}] show a trace for failed static checks\\ \hline

\end{tabular}

\vspace*{.5in}

\begin{tabular}{|l|p{1.4in}|}
\hline
\multicolumn{2}{|c|}{Options inherited from Java} \\
\hline
-Akey & \\ \hline
-bootclasspath {\it <path>}& See Java documentation. \\ \hline
-classpath {\it <path>}& location of class files \\ \hline
-cp {\it <path>}& location of class files\\ \hline
-d {\it <directory>} & location of output class files\\ \hline
-encoding {\it <encoding>} & \\ \hline
-endorsedirs {\it <dirs>} & \\ \hline
-extdirs {\it <dirs>} & \\ \hline
-deprecation & \\ \hline
-g & \\ \hline
-help & output help information\\ \hline
-implicit & \\ \hline
-J{\it <flag>} & \\ \hline
-nowarn & show only errors, no warnings \\ \hline
-proc & \\ \hline
-processor {\it <classes>} & \\ \hline
-processorpath {\it <path>} & where to find annotation processors\\ \hline
-s {\it <directory>} & location of output source files\\ \hline
-source {\it <release>} & the Java version of source files\\ \hline
-sourcepath {\it <path>} & location of source files\\ \hline
-target {\it <release>} & the Java version of the output class files\\ \hline
-X & Java non-standard extensions\\ \hline
-verbose & verbose output \\ \hline
-version & output (OpenJML) version\\ \hline
-Werror & treat warnings as errors \\ \hline

\end{tabular}
}
\caption{OpenJML options. See the text for more detail on each option.}
\label{Tab:Options}
\end{table}

\subsection{Specification files}
\label{SpecFiles}

JML specifications for Java classes (either source or binary) are written in files with a {\tt .jml} suffix or are written directly in the source {\tt .java} file.
When OpenJML needs specifications for a given class, it looks for a {\tt .jml} file on the specspath. If one is not found, OpenJML then looks for a {\tt .java}
file on the specspath. Note that this rule requires that source files (that have specifications you want to use) must be listed on the specspath. Note also that there 
need not be a source file; a {\tt .jml} file can be (and often is) used to provide specifications for class files.

Previous versions of JML had a more complicated scheme for constructing specifications for a class involving refinements, multiple specification files, and various prefixes. This complicated process is now deprecated and no longer supported.

[ TBD: some systems might find the first .java or .jml file on the specspath and use it, even if there were a .jml file later.]


\subsection{Annotations and the runtime library}

JML uses Java annotations as introduced in Java 1.6. Those annotation classes are in the package
{\tt org.jmlspecs.annotation}. In order for files using these annotations to be processed by Java,
the annotation classes must be on the classpath. They may also be required when a compiled Java program
that uses such annotations is executed. In addition, running a program that has JML runtime assertion
checks compiled in will require the presence of runtime classes that define utility functions used by the assertion checking code.

Both the annotation classes and the runtime checking classes are provided in a library named {\tt jmlruntime.jar}.  The distribution of OpenJML contains this library, as well as containing a
version of the library within {\tt openjml.jar}. When OpenJML is applied to a set of classes, by default it finds a version of the runtime classes and appends the location of the runtime classes
to the classpath.

You can prevent OpenJML from automatically adding {\tt jmlruntime.jar} to the classpath with the
option {\tt -noInternalRuntime}. If you use this option, then you will have to supply your own
annotation classes and (if using Runtime Assertion Checking) runtime utility classes on the classpath. You may wish to do this, for example, if you have newer versions of the annotation
classes that you are experimenting with. You could simply put them on the classpath, since they
would be in front of the automatically added classes and used in favor of default versions;
however, if you want to be sure that the default versions are not present, use the {\tt -noInternalRuntime} option.

The symptom that no runtime classes are being found at all is error messages that complain that
the {\tt org.jmlspecs.annotation} package is not found.


\subsection{Java properties and the {\tt openjml.properties} file}

OpenJML uses a number of properties that may be defined in the environment;
these properties are typically characteristics of the local environment that vary among different users or different installations. 
They can also be to set default values of options, so they do not need to be set on the command-line. An example is the file system location of a particular solver.

The tool looks for a file named {\tt openjml.properties} in several locations. It loads the
properties it finds in each of these, in order, so later definitions will supplant earlier ones.
\begin{itemize}\nospace
\item System properties, including those defined with {\tt -D} options on the command-line
\item On the system classpath
\item In the users home directory (the value of the Java property {\tt user.home}
\item In the current working directory (the value of the Java property {\tt user.dir}
\end{itemize}
[TBD: Check the above]

The properties that are currently recognized are these:
\begin{itemize}\nospace
%% TBD \item {\tt openjml.option.{\it <key>}}, where {\it <key>} is the name of a command-line option and the value is the value of the option as if it were specified on the command-line; values actually specified on the command-line override any specified in a properties file
\item {\tt openjml.defaultProver} - the value is the name of the prover to use by default
\item {\tt openjml.prover.{\it <name>}}, where {\it <name>} is the name of a prover, and
the value is the file system path to the executable to be invoked for that prover
\end{itemize}
[TBD: Check the above]

The distribution includes a file named {\tt openjml-template.properties} that contains stubs for all the recognized options.
You should copy that file, rename it as {\tt openjml.properties}, and edit it to reflect your system configuration.
(If you are an OpenJML developer, take care not to commit your system's {\tt openjml.properties} file into the OpenJML shared SVN repository.)

\subsection{OpenJML options}

There are many options that control or modify the behavior of OpenJML. Some of these are inherited from the Java compiler on 
which OpenJML is based. Options for the command-line tool are expressed as standard command-line options. In the Eclipse GUI, the values of options are set on a typical Eclipse preference or properties page for OpenJML.

The command-line options follow the style of the OpenJDK compiler --- they begin 
with a single hyphen and there are no two-hyphen versions.
OpenJML (but not OpenJDK) options that require a parameter may either use an = followed directly by the argument with no whitespace or 
may provide the argument as the subsequent entry of the argument list. 
For example, either \texttt{--racbin=output} or
\texttt{--racbin output} is permitted. If the argument is optional 
but present, the = form must be used. Values of options that contain whitespace must be quoted as appropriate for the operating system being used.

Options that are boolean in nature can be enabled and disabled by either
\begin{itemize}[noitemsep,nolistsep]
\item adding a prefix -no, as in \texttt{-showRacSource} and \texttt{-no-showRacSource}
\item or using the = form, as in \texttt{-showRacSource=true} and \texttt{-showRacSource=false}
\end{itemize}

\subsubsection{Options: Operational modes}
These operational modes are mutually exclusive.
\begin{itemize}[noitemsep,nolistsep]
\item \textbf{-jml} (default) : use the OpenJML implementation to process the listed files, including embedded JML comments and any .jml files
\item \textbf{-no-jml}: uses the OpenJML implementation to type-check and possibly compile the listed files, but ignores all JML annotations in those files
\item \textbf{-java}: processes the command-line options and files using only OpenJDK functionality. No OpenJML functionality is invoked. Must be the first option and overrides the others.
\end{itemize}

\subsubsection{Options: JML tools}
\label{OptionsTools}
The following mutually exclusive options determine which OpenJML tool is applied to the input files. They presume that the \texttt{-jml} mode is in effect.
\begin{itemize}
\item \textbf{-command} {\it <tool>} : initiates the given function; the value of {\it <tool>} may be one of {\tt check}, {\tt esc}, {\tt rac}, TBD.
The default is to use the OpenJML tool to do only typechecking of Java and JML in the source files.
\item \textbf{-check} : causes OpenJML to do only type-checking of the Java and JML in the input files (alias for \texttt{-command=check})
\item \textbf{-compile} : TBD
\item \textbf{-esc} : causes OpenJML to do (type-checking and) static checking of the JML specifications against the implementations in the input files  (alias for \texttt{-command=esc})
\item \textbf{-rac} : compiles the given Java files as OpenJDK would do, but with JML checks included for checking at runtime  (alias for \texttt{-command=rac})
\item \textbf{-doc} : executes javadoc but adds JML specifications into the javadoc output files.  (alias for \texttt{-command=doc})\textit{Not yet implemented.} 
\end{itemize}


\subsubsection{Options: Finding files and classes: class, source, and specs paths}
\label{OptionsPaths}

A common source of confusion is the various different paths used to find files, specifications and classes in OpenJML.
OpenJML is a Java application and thus a {\it classpath} is used to find the classes that constitute the OpenJML application;
but OpenJML is also a tool that processes Java files, so it uses a (different) classpath to find the files that it is processing. 
As is the case for other Java applications, a {\it <path>} contains a sequence of individual paths to folders or jar files, separated
by the path separator character (a semicolon on Windows systems and a colon on Unix and MacOSX systems).
You should distinguish the following:
\begin{itemize}
\item the classpath used to run the application: specified by one of
\begin{itemize}\nospace
\item the {\tt CLASSPATH} environment variable
\item the .jar file given with the {\tt java -jar} form of the command is used
\item the value for the {\tt -classpath} (equivalently, {\tt -cp}) option when OpenJML is run with the
{\tt java -cp openjml.jar org.jmlspecs.openjml.Main} command
\end{itemize}
This classpath is not of much concern to OpenJML, but is the classpath that Java users will be familiar with.
The value is implicitly given in the {\tt -jar} form of the command. The application classpath is explicitly given in the alternate form of the command,
and it may be omitted; if it is omitted, the value of the system property {\tt CLASSPATH} is used and it must contain the {\tt openjml.jar} library.

\item the classpath used by OpenJML. This classpath determines where OpenJML will find .class files for classes 
referenced by the {\tt .java} files it is processing. The classpath is specified by\\
\centerline{\tt -classpath {\it <path>}}
or \\
\centerline{\tt -cp {\it <path>}}
{\it after} the executable is named on the commandline.  That is,
\boxedexampleB{
java -jar openmjml.jar -cp {\it <openjml-classpath>} ...
}
or
\boxedexampleB{
java -cp openjml.jar org.jmlspecs.openjml.Main -cp {\it <openjml-classpath>} ...
}
If the OpenJML classpath is not specified, its value is the same as the application classpath.

\item the OpenJML sourcepath - The sourcepath is used by OpenJML as the list of locations in which to find {\tt .java} 
files that are referenced by the files being processed. For example, if a file on the command-line, say {\tt T.java},
refers to another class, say {\tt class U}, that is not listed on the command-line, then {\tt U} must be found.  OpenJML (just as is done by the Java compiler) will look for a source file for {\tt U} in the sourcepath and a class file for {\tt U} in the classpath.
If both are found then TBD.

The OpenJML sourcepath is specified by the {\tt -sourcepath {\it <path>}} option. If it is not specified, the value for the sourcepath is taken to be the same as the OpenJML classpath.

In fact, the sourcepath is rarely used.  Users often will specify a classpath containing both {\tt .class} and {\tt .java} files; by 
not specifying a sourcepath, the same path is used for both {\tt .java} and {\tt .class} files. This is simpler to write, but does mean
that the application must search through all source and binary directories for any particular source or binary file.

\item the OpenJML specspath - The specspath tells OpenJML where to look for specification ({\tt .jml}) files. It is specified with the {\tt -spacspath {\it <path>}} option. If it is not specified, the value for the specspath is the same as the value for the sourcepath.  In addition, by default, the specspath
has added to it an internal library of specifications.  These are the existing (and incomplete) specifications of the Java standard library classes.

The addition of the Java specifications to the specspath can be disabled by using the {\tt -noInternalSpecs} option.  For example. if you
have your own set of specification files that you want to use instead of the internal library, then you should use the {\tt -noInternalSpecs} option and a {\tt -specspath} option with a path that includes your own specification library.

Note also that often source ({\tt .java}) files contain specifications as well. Thus, if you are specifying a specspath yourself, you should
be sure to include directories containing source files in the specspath; this rule also includes the {\tt .java} files that appear on the 
command-line: they also should appear on the specspath.

TBD - describe what happens if the above guidelines are not followed. (Can we make this more user friendly).

\end{itemize}

\paragraph{The {\tt -noInternalSpecs} option.} As described above, this option turns off the automatic adding of the internal specifications library to the specspath. If you use this option, it is your responsibility to provide an alternate specifications library for the standard
Java class library. If you do not you will likely see a large number of static checking warnings when you use Extended Static Checking to check the implementation code against the specifications.

The internal specifications are written for programs that conform to Java 1.7.  [ TBD - change this to adhere to the {\tt -source} option?] 
[TBD - what about the specs in jmlspecs for different source levels.]

\subsubsection{Options: OpenJML options applicable to all OpenJML tools }
\label{OptionsJML}

\begin{itemize}
\item \textbf{-dir} \textit{<folder>} : abbreviation for listing on the command-line all of the .java files in the given folder, and its subfolders; if the argument is a file, use it as is
\item \textbf{-dirs} : treat all subsequent command-line arguments as if each were the argument to \texttt{-dir}
\item \textbf{-specspath} \textit{<path>} : defines the specifications path, cf. section TBD
\item \textbf{-keys} \textit{<keys>} : the argument is a comma-separated list of options JML keys (cf. section TBD)
\item \textbf{-strictJML} : warns about an OpenJML extensions to standard JML
\end{itemize}

\begin{itemize}
\item \textbf{-nullableByDefault} : sets the global default to be that all declarations are implicitly \texttt{@Nullable}
\item \textbf{-nonnullByDefault} : sets the global default to be that all 
declarations are implicitly \texttt{@NonNull} (the default)
\item \textbf{-purityCheck} : turns on (default is on) purity checking (recommended since the Java library specifications are not complete for \texttt{@Pure} declarations)
\end{itemize}

\subsubsection{Options: Extended Static Checking}
\label{OptionsESC}

These options apply only when performing ESC:
\begin{itemize}
\item \textbf{-prover} \textit{<prover>} : the name of the prover to use: one of z3\_4\_3, cvc4, yices2
\item \textbf{-exec} \textit{<file>} : the path to the executable corresponding to the given prover
\item \textbf{-boogie} : enables using boogie (-prover option ignored; -exec must specify the Z3 executable for Boogie to use)
\item \textbf{-method} \textit{<methodlist>} : a comma-separated list of method names to
check (default is all methods in all listed classes)
\item \textbf{-exclude} \textit{<methodlist>} : a comma-separated list of method names to exclude from checking
\item \textbf{-checkFeasibility} \textit{<where>} : checks feasibility of the program at various points --- a comma-separated list of
one of \texttt{none}, \texttt{all}, \texttt{exit} [TBD, finish list,  give default]
\item \textbf{-escMaxWarnings} \textit{<int>} : the maximum number of assertion violations to look for; the argument is either a positive integer or \texttt{All}; the default is \texttt{All}
\item \textbf{-counterexample} : prints out a counterexample for failed proofs
\item \textbf{-trace} : prints out a counterexample trace for each failed assert (includes -counterexample)
\item \textbf{-subexpressions} : prints out a counterexample trace with model values for each subexpression (includes -trace)
\end{itemize}

\subsection{Options: Runtime Assertion Checking}
\label{OptionsRAC}

These options apply only when doing RAC:
\begin{itemize}
\item \textbf{-showNotExecutable} : warns about the use of features that are not executable (and thus ignored); turn off with \texttt{-no-shownotExecutable}
\item \textbf{-showRacSource} : enables including source code information in RAC error messages (default is enabled; disable with \texttt{-no-showRacSource})
\item \textbf{-racCheckAssumptions} : enables checking \texttt{assume} statements as if they were asserts (default is enabled; disable with\texttt{-no-racCheckAssumptions})
\item \textbf{-racJavaChecks} : enables performing JML checking of violated Java features (which will just proceed to throw an exception anyway) (default is enabled; disable with \texttt{-no-racJavaChecks})
\item \textbf{-racCompileToJavaAssert} : compile RAC checks using Java asserts (which must then be enabled using \texttt{-ea}) (default is disabled; disable with \texttt{-no-racCompileToJavaAssert})
\end{itemize}

\subsubsection{Options: Version of Java language or class files}

\begin{itemize}
\item \textbf{-source} {\it <level>} : this option specifies the Java version of the source files, with values of {\tt 1.4}, {\tt 1.5}, {\tt 1.6}, {\tt 1.7}... or {\tt 4}, {\tt 5}, {\tt 6}, {\tt 7}, ... . This controls whether some syntax features  
(e.g. annotations, extended for-loops, autoboxing, enums) are permitted. The default is the most recent version
of Java, in this case 1.7.  Note that the classpath should include the Java library classes that
correspond to the source version being used.

\item \textbf{-target} {\it <level>} : this option specifies the Java version of the output class files
\end{itemize}


\subsubsection{Options: Other Java compiler options applicable to OpenJML}

All the OpenJDK compiler options apply to OpenJML as well. The most commonly used or important OpenJDK options are listed here.

These options control where output is written:
\begin{itemize}
\item \textbf{-d} {\it <dir>} : specifies the directory in which output class files are placed; the directory must already exist
\item \textbf{-s} {\it <dir>} : specifies the directory in which output source files are placed; such as those produced by annotation processors; the directory must already exist
\end{itemize}

These are Java options relevant to OpenJML whose meaning is unchanged in OpenJML. 
\begin{itemize}[noitemsep,nolistsep]
\item \textbf{-cp} or \textbf{-classpath}: the parameter gives the classpath to use to find unnamed but referenced class files (cf. section TBD)
\item \textbf{-sourcepath}: the parameter gives the sequence of directories in which to find source files for unnamed but referenced classes (cf. section TBD)
\item \textbf{-deprecation}: enables warnings about the use of deprecated features (applies to deprecated JML features as well)
\item \textbf{-nowarn}: shuts off all compiler warnings, \textit{including the static check warnings produced by ESC}
\item \textbf{-Werror}: turns all warnings into errors, including JML (and static check) warnings
\item \textbf{@\textit{filename}}: the given \textit{filename} contains a list of arguments
\item \textbf{-source}: specifies the Java version to use (default 1.7)
\item \textbf{-verbose}: turn on Java verbose output
\item \textbf{-Xprefer:source} or \textbf{-Xprefer:newer}: when both a .java and a .class file are present, 
whether to choose the .java (source) file or the file that has the more recent modification time [ TBD - check that this works ]
\item \textbf{-stopIfParseErrors}: if enabled (disabled by default), processing stops after parsing if there are any parsing errors (TBD - check this, describe the default)
\end{itemize}




Other Java options, whose meaning and use is unchanged from javac:
\begin{itemize}
\item \textbf{@}\textit{<filename>} : reads the contents of \textit{<filename>} as a sequence of command-line arguments (options, arguments and files)
\item \textbf{-Akey}
\item \textbf{-bootclasspath}
\item \textbf{-encoding}
\item \textbf{-endorsedirs}
\item \textbf{-extdirs}
\item \textbf{-g}
\item \textbf{-implicit}
\item \textbf{-J}
\item \textbf{-nowarn} : only print errors, not warnings, \textit{including not printing static check warnings}
\item \textbf{-Werror} : turns all warnings into errors
\item \textbf{-X}... : Java's extended options
\end{itemize}

These Java options are discussed elsewhere in this document:
\begin{itemize}
\item \textbf{-cp} \textit{<path>} or \textbf{-classpath} \textit{<path>} : section \ref{OptionsPaths}
\item \textbf{-sourcepath} \textit{<path>} : section \ref{OptionsPaths}
\item \textbf{-verbose} : section \ref{OptionsDebugging}
\item \textbf{-source} :  
\item \textbf{-target} :  
\end{itemize}


\subsubsection{Options: Information and debugging}
\label{OptionsDebugging}
These options print summary information and immediately exit (despite the presence of other command-line arguments):
\begin{itemize}
\item \textbf{-help} : prints out help information about the command-line options
\item \textbf{-version} : prints out the version of the OpenJML tool software
\end{itemize}
The following options provide different levels of verboseness. If more than one is specified, the last one present overrides earlier ones.
\begin{itemize}
\item \textbf{-quiet} : no informational output, only errors and warnings
\item \textbf{-normal} : (default) some informational output, in addition to errors and warnings
\item \textbf{-progress} : prints out summary information as individual files are processed (includes -normal)
\item \textbf{-verbose} : prints out verbose information about the Java processing in OpenJDK (does not include other OpenJML information)
\item \textbf{-jmlverbose} : prints out verbose information about the JML processing (includes -verbose and -progress)
\item \textbf{-jmldebug} : prints out (voluminous) debugging information (includes -jmlverbose)
\item \textbf{-verboseness} {\it <int>} : sets the verboseness level to a value from 0 - 4, corresponding to -quiet, -normal, -progress, -jmlverbose, -jmldebug
\end{itemize}

Other debugging options:
\begin{itemize}
\item \textbf{-show} : prints out rewritten versions of the Java program files for informational and debugging purposes
\item \textbf{-showNotImplemented} : prints warnings about JML features that are ignored because they are not implemented; 
disable with \texttt{-no-showNotImplemented}; the default is disabled.
\end{itemize}

An option used primarily for testing:
\begin{itemize}[noitemsep,nolistsep]
\item \textbf{-jmltesting}: adjusts the output so that test output is more stable
\end{itemize}

\subsection{Options related to Static Checking}
\begin{itemize}
\item -counterexample
\item -trace
\item -subexpressions
\item -method
\end{itemize}

\subsection{Options related to parsing and typechecking}
\begin{itemize}
\item -Werror
\item -nowarn
\item -stopIfParseError
\item -checkSpecsPath
\item -purityCheck
\item -nonnullbydefault
\item -nullablebydefault
\item -keys
\end{itemize}

\subsection{Java options related to annotation processing}
\begin{itemize}
\item -proc
\item -processor
\item -processorpath
\end{itemize}

\subsection{Other JML Options}
\begin{itemize}
\item -roots
\end{itemize}


\paragraph{General options}
\begin{itemize}[noitemsep,nolistsep]
\item \textbf{-dir}: Indicates that its argument is a directory. All the .java and .jml files in the directory and its subdirectories are processed. (TBD - is this necessary?)
\item \textbf{-dirs}: Indicates that all subsequent command-line arguments are directories, to be processed as for \texttt{-dir}, until an argument is reached that begins with a hyphen. 
\item \textbf{-specspath}: the parameter gives the sequence of directories in which to find .jml specification files for unnamed but referenced classes (cf. section TBD)
\item \textbf{-checkSpecsPath}: if enabled (the default), warns about \texttt{specspath} elements that do not exist
\item \textbf{-keys}: comma-separated list of the optional JML comment keys to enable (empty by default)
\item \textbf{-strictJML}: (disabled by default) warns about the use of any OpenJML extensions to standard JML; disable with -no-strictJML
\item \textbf{-showNotImplemented}: (disabled by default) warns about the use of features that are not implemented; disable with -no-showNotImplemented
\end{itemize}


\textit{This section will be completed later.} %% TBD



\chapter{The Eclipse Plug-in}
\label{EclipsePlugin}

Since OpenJML operates on Java files, it is natural that it be integrated into the Eclipse IDE for Java.
OpenJML provides a conventional Eclipse plug-in that encapsulates the OpenJML command-line tool and integrates it
with the Eclipse Java development environment. The plug-in also provides GUI functionality for working with JML specifications.

\section{Installation and System Requirements}
Your system must have the following:
\begin{itemize}
\item A Java 1.7 JRE as described in section \ref{Running}. This must be the JRE in use in the environment in which Eclipse is invoked. If you start Eclipse by a command in a shell, it is straightforward to make sure that the correct Java JRE is defined in that shell.  However, if you start Eclipse by, for example, double-clicking a desktop icon, then you must ensure that the Java 1.7 JRE is set by the system at startup.
\item Eclipse 4.2 or later. The OpenJML plug-in is currently tested for the Juno, Kepler, and Luna releases of Eclipse.
\item One or more SMT solvers, if the static checking functionality will be used.
\end{itemize}

Installation of the plug-in follows the conventional Eclipse procedure.
\begin{itemize}
\item Invoke the "Install New Software" dialog under the Eclipse "Help" menubar item.
\item "Add" a new location, giving the URL \url{http://jmlspecs.sourceforge.net/openjml-updatesite} and some name of your choice (e.g. OpenJML).
\item Select the "OpenJML" category and push "Next"
\item Proceed through the rest of the wizard dialogs to install OpenJML.
\item Restart Eclipse when asked to obtain full functionality.
\end{itemize}

If the plug-in is successfully installed, a yellow coffee cup (the JML icon) will appear in the toolbar (along with other menubar/toolbar items).
The installation will fail (without obvious error messages), if the underlying Java VM is not a suitable Java 1.7 VM.

\section{GUI Features}

Note that the JML logo is a JML-decorated yellow coffee cup; this logo is associated with various GUI elements.
 
\subsection{Commands}
The OpenJML plug-in adds a number of commands. These are visible in the Preferences>>General>>Keys dialog. All the OpenJML commands explicitly added by the OpenJML plug-in are in the 'JML' category. Some commands are automatically added by Eclipse and are in other categories; for example, Eclipse automatically adds commands to open each individual Preference Page and each kind of View You can sort the table of commands by category and you can filter the table, in order to show just those commands related to JML. Also, this dialog allows binding a keyboard key-combination to a command, as is the case for all Eclipse commands.

The commands are listed in Table~\ref{Tab:commands}, with forward references to more detailed discussion.
\begin{itemize}
\item Add to JML specs path (\S\ref{TBD}). Allows editing the specspath 
\item Clear All Results (\S\ref{TBD}). Deletes all results of static checking operations 
\item Clear Selected Results (\S\ref{TBD}). Deletes some of the results of static checking 
\item Delete JML Markers (\S\ref{TBD}). Deletes all JML markers and highlighting on selected resources 
\item Disable JML on the project (\S\ref{TBD}). 
\item Edit JML Source/Specs Paths (\S\ref{TBD}). 
\item Enable JML on the project (\S\ref{TBD}). 
\item Generate JML doc (\S\ref{TBD}). 
\item insert \bs result (\S\ref{TBD}). 
\item insert .... (\S\ref{TBD}). 
\item Open a Specifications Editor (\S\ref{TBD}). 
\item Open the ESC Results View (\S\ref{TBD}). 
\item RAC - ... (\S\ref{TBD}). 
\item Remove from JML specspath (\S\ref{TBD}). 
\item Rerun Static Check (\S\ref{TBD}). 
\item Show Counterexample (\S\ref{TBD}). 
\item Show Counterexample Value (\S\ref{TBD}). 
\item Show Detailed Proof Attempt Information (\S\ref{TBD}). 
\item Show ESC Result Information (\S\ref{TBD}). 
\item Show JML paths (\S\ref{TBD}). 
\item Show JML Specifications (\S\ref{TBD}). 
\item Static Check (ESC) (\S\ref{TBD}). 
\item Typecheck JML (\S\ref{TBD}). Performs syntax, parsing and typechecking on selected projects, folders, and files.
\item Show In ... (\S\ref{TBD}). 
\item Show View (OpenJML Static Checks) (\S\ref{TBD}). 
\item Show View (OpenJML Trace) (\S\ref{TBD}). 
\item Preferences (OpenJML > OpenJML Solvers) (\S\ref{TBD}). Opens the OpenJML Solvers subpage.
Preferences (OpenJML) (\S\ref{TBD}). Opens the OpenJML Preferences page.
\end{itemize}



\subsection{Menubar additions}
The main Eclipse menubar contains an additional menu titled 'JML'. It contains the following submenu items; typically the action for a menu item is the similarly named command.
Menu items are also added in the following Context menus (context menus are available by right-clicking):
\begin{itemize}[noitemsep,nolistsep]
\item Context menu on elements of the Eclipse Package Explorer View, Project Explorer View, and Navigator View
\item Context menu on elements of the Outline View
\item Context menu within an editor
\item ... Problem View...
\item ... OpenJML Proof view ...
\end{itemize}
Individual menu items may be disabled when they are not applicable. For instance, some items are enabled only when something is selected, some only when exactly one apporpriate item is selected, some only when a method is selected, etc.
%\begin{itemize}
%\item Typecheck JML - executes the 'Typecheck JML' command
%\item TBD - MORE ...
%\end{itemize}

\subsection{Toolbar additions}
The OpenJML plug-in adds three toolbar items; clicking the toolbar item executes a corresponding command.
\begin{itemize}
\item the JML coffee cup logo - executes the 'Typecheck JML' command
\item ESC - executes the 'Static Check (ESC)' command
\item RAC - executes the 'RAC - compile selected' command
\end{itemize}

\subsection{OpenJML Problems, Markers and highlights}

TBD

\subsection{OpenJML console and error log}
The plug-in adds an additional type of console. Eclipse has a Console View that manages the consoles for various plug-ins. The OpenJML Console is created automatically when output is generated. General informational output is sent to the console; errors and warnings are placed in the console and shown in pop-up dialog boxes. Egregious errors are also logged in the Eclipse Error Log.

\textit{Note: Currently the OpenJML Console is not listed in the list of types of Consoles available in Console View menu.}

\subsection{Preferences}

The plug-in adds dialogs for setting OpenJML options. These are workspace preferences, affecting all projects in the workspace. There are two Preference pages:
\begin{itemize}[noitemsep,nolistsep]
\item A top-level page named 'OpenJML' found in the top-level list of Eclipse preference pages. This page allows setting options that would otherwise be set on the command-line. These is also one UI option, named 'UI verbose', that enables verbose output to the OpenJML console about actions within the UI code. 
\item A sub-page named 'Solvers'. (Click the turnstile next to 'OpenJML' in the list of Preference pages to see the subpage). These preferences enable registering SMT solvers and setting the default solver to use.
\end{itemize}

There are currently no project-level preferences within the OpenJML plug-in.

\subsection{Editor embellishments}

TBD - fill out

\begin{itemize}
\item counterexample hovers
\item quick fix proposals
\item contet-sensitive completions
\item insertions
\end{itemize}

\subsection{OpenJML Views}

TBD

\subsection{Help}
There is a 'JML' entry in the table of contents under the Eclipse Help menu item. It provides an online user guide to JML and OpenJML.

\textit{Note - the current information available under Help is outdated and will be replaced by this manual.}

\subsection{Other GUI elements}
\begin{itemize}
\item OpenJML decoration - A decoration is applied to names of projects in the Package Explorer View for which OpenJML has been enabled (cf. \S\ref{TBD}). The decoration is a miniature JML logo on the upper-right of the folder icon, covering the 'J' symbol that indicates an Eclipse Java project.
\item The plug-in defines a JML Nature and a JML Builder. The Nature is associated with a project precisely when JML is enabled for the project. The Builder performs automatic type-checking.  (cf. \S\ref{TBD})
\item \texttt{.jml} suffix. The plug-in adds a content type that associates the \texttt{.jml} filename suffix with the Java editor. This makes the Java editor the default editor for .jml files. 

\textit{Note - so we still get spurious errors on jml files?} % TBD

\item Internationalization. TBD...
\item classpath intializer. TBD ...
\item definition as an Eclipse project. TBD ...
\item Open JML Perspective. TBD ...
\end{itemize}

\chapter{OpenJML tools}

\section{Parsing and Type-checking}

The foundational function of OpenJML is to parse and check the well-formedness of JML annotations in the context of the associated Java program.
Such checking includes conventional type-checking and checking that names are used consistently with their visibility and purity status. 

A set of Java files with JML annotations is type-checked with the command
\boxedexampleZ{java -jar \textit{\$INSTALL}/openjml.jar -check \textit{options} \textit{files}}
or
\boxedexampleZ{java -jar \textit{\$INSTALL}/openjml.jar \textit{options} \textit{files}}
since \texttt{-check} is the default action.
The equivalent action in the Eclipse plug-in is the 'Typecheck JML' command, available through the toolbar or menu actions.
Any \texttt{.jml} files are checked when the associated \texttt{.java} file is created. Only \texttt{.java} files either listed on the command-line or contained in folders listed on the command-line are certain to be checked. Some checking of other files may be performed where references are made to classes or methods in those non-listed files.

A key concept to understand is how class files, source files, and specification files are found and used by the OpenJML tool. This
process is described in the following subsection. The command-line options relevant to parsing and type-checking are discussed in the subsequent subsection.

\subsection{Classpaths, sourcepaths, and specification paths in OpenJML}

When a Java compiler compiles source files, it considers three types of files:
\begin{itemize}[noitemsep,nolistsep]
\item Source files listed on the command-line
\item Other source files referenced by those listed on the command-line, but not on the command-line themselves
\item Already-compiled class files
\end{itemize}
The OpenJML tool considers the same files, but also needs
\begin{itemize}[noitemsep,nolistsep]
\item Specification files associated with classes in the program
\end{itemize}

The OpenJML tool behaves in a way similar to a typical Java compiler, 
making use of three directory paths - the classpath, the sourcepath, and the specspath.  
These paths are standard lists of directories or jar files, 
separated either by colons (Unix) or semicolons (Windows).  
Java packages are subdirectories of these directories.

\begin{itemize}[noitemsep,nolistsep]
\item \texttt{classpath}: The OpenJML classpath is set using one of these alternatives, in priority order:
\begin{itemize}[noitemsep,nolistsep]
\item As the argument to the OpenJML command-line option \texttt{-classpath}
\item As the value of the Java property \texttt{org.jmlspecs.openjml.classpath}
\item As the value of the system environment variable \texttt{CLASSPATH}
\end{itemize}
\item \texttt{sourcepath}: The OpenJML sourcepath is set using one of these alternatives, in priority order:
\begin{itemize}[noitemsep,nolistsep]
\item As the argument of the OpenJML command-line option \texttt{-sourcepath}
\item As the value of the Java property \texttt{org.jmlspecs.openjml.sourcepath}
\item As the value of the OpenJML classpath (as determined above)
\end{itemize}
\item \texttt{specspath}: The OpenJML specifications path is set using one of these alternatives, in priority order:
\begin{itemize}[noitemsep,nolistsep]
\item As the argument of the OpenJML command-line option \texttt{-specspath}
\item As the value of the Java property \texttt{org.jmlspecs.openjml.specspath}
\item As the value of the OpenJML sourcepath (as determined above)
\end{itemize}
\end{itemize}

Note that with no command-line options or Java properties set, the result is simply that the system CLASSPATH is used for all of these paths. A common practice is to simply use a single directory path, specified on the command-line using \texttt{-classpath}, for all three paths.

The paths are used as follows to find relevant files:
\begin{itemize}[noitemsep,nolistsep]
\item Source files listed on the command-line are found directly in the file system. 
If the command-line element is an absolute path to a \texttt{.java} file, it is looked up in the file system as an absolute path; 
if the command-line element is a relative path, the file is found relative to the current working directory.
\item Classes that are referenced by files on the command-line or transitively by other classes in the program, can be found in one of two ways:
\begin{itemize}[noitemsep,nolistsep]
\item The source file for the class is sought as a sub-file of an element of the \texttt{sourcepath}.
\item The class file for the class is sought as a sub-file of an element of the \texttt{classpath}.
\end{itemize}
If there is both a sourcefile and a classfile present, then
\begin{itemize}[nolistsep,noitemsep]
\item if the option \texttt{-Xprefer:source} is present, the sourcefile is always recompiled
\item if the option \texttt{-Xprefer:newer} is present, the sourcefile is recompiled only if its modification timestamp is newer than that of the class file.
\end{itemize}
The default is to use the newer of the source or class files.
\end{itemize}





\subsection{Command-line options for type-checking}

The following command line options are particularly relevant to type-checking.
\begin{itemize}[noitemsep,nolistsep]
\item \textbf{-nullableByDefault}: sets the global default to be that all variable, field, method parameter, and method return type declarations are implicitly \texttt{@Nullable}
\item \textbf{-nonnullByDefault}: sets the global default to be that all variable, field, method parameter, and method return typedeclarations are implicitly \texttt{@NonNull} (the default)
\item \textbf{-purityCheck}: enables (default on) checking for purity; disable with \texttt{-no-purityCheck}
\item \textbf{-internalSpecs}: enables (default on) using the built-in library specifications; disable with -\texttt{no-internalSpecs}
\item \textbf{-internalRuntime}: enables (default on) using the built-in runtime library; disable with -\texttt{no-internalRuntime}
\end{itemize}


\section{Static Checking and Verification}
\textit{This section will be added later.} %% TBD

\subsection{Options specific to static checking}
\begin{itemize}[noitemsep,nolistsep]
\item \textbf{-prover \textit{prover}}: the name of the prover to use: one of z3\_4\_3, yices2 [TBD: expand list]
\item \textbf{-exec \textit{path}}: the path to the executable corresponding to the given prover
\item \textbf{-boogie}: enables using boogie (-prover option ignored; -exec must specify the Z3 executable)
\item \textbf{-method \textit{methodlist}}: a comma-separated list of method names to check (default is all methods in all listed classes) [TBD - describe wildcards and fully 
\item \textbf{-exclude \textit{methodlist}}: a comma-separated list of method names to exclude from checking
\item \textbf{-checkFeasibility \textit{where}}: checks feasibility of the program at various points:
one of \texttt{none}, \texttt{all}, \texttt{exit} [TBD, finish list, give default]
\item \textbf{-escMaxWarnings \textit{int}}: the maximum number of assertion violations to look for; the argument is either a positive integer or \texttt{All} (or equivalently \texttt{all}, default is \texttt{All})
\item \textbf{-trace}: prints out a counterexample trace for each failed assert
\item \textbf{-subexpressions}: prints out a counterexample trace with model values for each subexpression
\item \textbf{-counterexample} or \textbf{-ce}: prints out counterexample information
\end{itemize}

\section{Runtime Assertion Checking}
\textit{This section will be added later.} %% TBD

\subsection{Options specific to runtime checking}
\begin{itemize}[noitemsep,nolistsep]
\item \textbf{-showNotExecutable}: warns about the use of features that are not executable (and thus ignored)
\item \textbf{-racShowSource}: includes source location in RAC warning messages [ TBD: default? ]
\item \textbf{-racCheckAssumptions}: enables (default on [TBD - is this default correct?]) checking \texttt{assume} statements as if they were asserts
\item \textbf{-racJavaChecks}: enables (default on) performing JML checking of violated Java features (which will just proceed to throw an exception anyway)
\item \textbf{-racCompileToJavaAssert}: (default off) compile RAC checks using Java asserts (which must then be enabled using \texttt{-ea}), instead of using \texttt{org.jmlspecs.utils.JmlAssertionFailure}
\item \textbf{-racPreconditionEntry}: (default off) enable distinguishing internal Precondition errors from entry Precondition errors, appropriate for automated testing; compiles code to generate JmlAssertionError exceptions
(rather than RAC warning messages)[TBD - should this turn on -racCheckAssumptions?] 
\end{itemize}

\section{Inferring Specifications}
\label{sec:inferring-specifications}

OpenJML is able to infer specifications for many methods. Fundamentally, OpenJML does this statically, by using the strongest postcondition predicate transformer semantics (sp). Unfortunately, although the specifications produced by sp are precise, they are often exponential with respect to the code they describe. One innovation of our tool is that we apply a principled technique for making these specifications more concise. The documentation contained in this section details the options for utilizing these features within OpenJML. 

The technical details pertaining to the specification inference technique of OpenJML may be found in the following technical report: \url{http://www.cs.ucf.edu/~leavens/tech-reports/UCF/CS-TR-18-01/TR.pdf}


\subsection{Options specific to specification inference}

\begin{itemize}[noitemsep,nolistsep]
    \item \textbf{-infer}: Instructs OpenJML to attempt to infer specifications. 
    \item \textbf{-infer-preconditions}: If no preconditions are specified, by default OpenJML will use the precondition of \texttt{\textbf{true}}. Setting this to false disables this behavior. 
    \item \textbf{-infer-tag}: By default, inferred specifications are given the specification key \texttt{INFERRED}. Setting this option to \texttt{\textbf{false}} disables this behavior. 
    \item \textbf{-infer-persist}: Specifications may be persisted in two ways: either to an independent \texttt{.jml} file or \textit{woven} into the file containing the method being inferred. If the option is specified as \texttt{-infer-persist=java} the specifications will be woven. By default (or if \texttt{jml} is specified) the specs will be written to a separate \texttt{.jml} file.

    \item \textbf{-infer-persist-path}: When writing specs to \texttt{.jml} files, this option controls the location the files are written to. If nothing is specified, the spec will be written to the directory containing the file containing the methods being inferred. If a path is specified, the specs will be written out to their package, using the supplied path as a base. If \texttt{-specspath} is specified (and \texttt{-infer-persist-path} is not), the base path will be the one given by \texttt{-specspath}.

    \item \textbf{-infer-max-depth}: The size of the control flow graph (CFG) of a program influences how long specification inference will take. By default, OpenJML will refuse to infer methods with a CFG larger than 300 nodes. You can specify an integer value here to override this behavior.

    \item \textbf{-infer-timeout}: By default, OpenJML will skip methods that take longer than 300 seconds to infer. To change this behavior, specify an integer value for this parameter. 
    
    \item \textbf{-infer-debug}: Enables several GUI tools for debugging specification inference. Intended only for developers. 

\textit{Note:} Specification inference supports the \texttt{-method} options supported by other tools (such as ESC) for selecting the methods to be inferred. 


\end{itemize}


\section{Generating Documentation}
\textit{This section will be added later.} %% TBD

\section{Generating Specification File Skeletons}
\textit{This section will be added later.} %% TBD

\section{Generating Test Cases}
\textit{This section will be added later.} %% TBD

\section{Limitations of OpenJML's implementation of JML}
Currently OpenJML does not completely implement JML. The differences are explained in the following subsections.

\subsection{model import statement}
OpenJML translates a JML model import statement into a regular Java import statement [TBD - check this].
COsequently, names introduced in a model import statement are visible in both Java code and JML annotations.
This has consequences in the situation in which a name is imported both through a Java import and a JML model import.
Consider the following examples of involving packages \texttt{a} and \texttt{b}, each containing a class named 
\texttt{X}.

In these two examples,
\boxedexampleZ{
import a.X; //@ model import b.X;
}
\boxedexampleS{
import a.*; //@ model import b.*;
}
the class named \texttt{X} is imported by both an import statement and a model import statement. In JML, the use of \texttt{X} in
Java code unambiguously refers to \texttt{a.X}; the use of \texttt{X} in JML annotations is ambiguous. However, in OpenJML,
the use of \texttt{X} in both contexts will be identified as ambiguous.

In
\boxedexampleZ{
import a.*; //@ model import b.X;
}
a use of \texttt{X} in Java code refers to \texttt{a.X} and a use in JML annotations refers to \texttt{b.X}.
However, in OpenJML, both uses will mean \texttt{b.X}.

However,
\boxedexampleZ{
import a.X; //@ model import b.*;
}
is unproblematic. Both JML and OpenJML will interpret \texttt{X} as \texttt{a.X} in both Java code and JML annotations.

TBD - more to be said about .jml files

\subsection{purity checks and system library annotations}

JML requires that methods that are called within JML annotations must be pure methods (cf. section TBD). OpenJML does implement
a check for this requirement. However, to be pure, a method must be annotated as such by either \texttt{/* pure */} or 
\texttt{@Pure}. A user should insert such annotations where appropriate in the user's own code. However, many system libraries 
still lack JML annotations, including indications of purity. Using an unannotated library call within JML annotation will provoke a warning from OpenJML. Until the system libraries are more thoroughly annotated, users may wish to use the \texttt{-no-purityCheck} option to turn off purity checking.

\subsection{TBD - other unimplemented features}


\chapter{Using OpenJML and OpenJDK within user programs}

The OpenJML software is available as a library so that Java and JML programs can be manipulated 
within a user's program. The developer needs only to include the {\tt openjml.jar} library on 
the classpath when compiling a program and to call methods through the public API as described in this chapter.
The public API is implemented in the interface {\tt org.jmlspecs.openjml.IAPI}; it provides the ability to
\begin{itemize}
\item perform compilation actions as would be executed on the command-line
\item parse files or Strings containing Java and JML source code, producing parse trees
\item print parse trees
\item walk over parse trees to perform user-defined actions
\item type-check parse trees (both Java and JML checking)
\item perform static checking
\item compile modules with run-time checks
\item emit javadoc documentation with JML annotations
\end{itemize}
The sections of this chapter describe these actions and various concepts needed to perform them correctly.

CAUTION: OpenJML relies on parts of the OpenJDK software that are labeled as internal, non-public and subject to
change. Correspondingly, some of the OpenJML API may change in the future. The definition of the API class is 
intended to provide a buffer against such changes. However, the names and functionality of OpenJDK classes (e.g.,
the {\tt Context} class in the next section) could change.

\paragraph{List classes}
CAUTION \#2: The OpenJDK software uses its own implementation of Lists, namely {\tt com.sun.tools.javac.util.List}. 
It is a different implementation than {\tt java.util.List}, with a different interface. Since one or the other may
 be in the list of imports, the use of {\tt List} in the code may not clearly indicate which type of List is being
 used. Error messages are not always helpful here. Users should keep these two types of List in mind to avoid 
 confusion.

\paragraph{Example source code}
The subsections that follow contain many source code examples. 
Small source code snippets are shown in in-line boxes like this:
\boxedexample{ // A Java comment }
Larger examples are shown as full programs. These are followed by a box of text with a gray background that contains the output expected if the program is run (if the program is error-free) or compiled (if there are compilation errors).
Here is a ``Hello, world'' example program:
\demo{DemoHelloWorld}

\noindent
All of these full-program example programs are working, tested examples.
They are available in the {\tt demos} directory of the OpenJML source code. The opening comment
line (as well as the class name) of the example text gives the file name.

The full programs presume an appropriate environment. In particular, they expect the following
\begin{itemize}
\item the current working directory is the {\tt demos} directory of the OpenJML source distribution
\item the Java {\tt CLASSPATH} contains the current directory and a release version of the
OpenJML library ({\tt openjml.jar}). For example, if the demos directory is the current working directory and a copy of {\tt openjml.jar} is in the 
{\tt demos} directory, then the {\tt CLASSPATH} could be set as {\tt ``.;openjml.jar''} (using the ; on Windows, a : on Mac and Linux)
\end{itemize}
Note that the examples often use other files that are in subdirectories of the {\tt demos} directory.
\boxedexampleZ{ % FIXME - this does not translate well to HTML
// bash commands to compile and run the DemoHelloWorld example
\begin{tabular}{ll}
cd OpenJML/demos     &\# Alter this to match your local installation \\
export CLASSPATH=''.;openjml.jar''     &\# Use a : instead of ; on Unix or Mac \\
                                       &\# Copy openjml.jar to the demo directory \\
javac DemoHelloWorld.java              &\# Be sure java tools from a 1.7 JDK  \\
java DemoHelloWorld                    &\# are on the PATH
\end{tabular}
}

\section{Concepts}

\subsection{Compilation Contexts}
All parsing and compilation activities within OpenJML are performed with respect to a {\em compilation context},
implemented in the code as a {\tt com.sun.tools.javac.util.Context} object. There can be more than one Context at a given time,
though this is rare. A context holds all of the symbol tables and cached values that represent the source code created
in that context. 

There is little need for the user to create or manipulate
Contexts. However it is essential that items created in one Context not be used in another context. There is no check for such misuse, but the subsequent actions are likely to fail. For example, a Context contains interned versions of the names of
source code identifiers (as {\tt Name}s). Consequently an identifier parsed in one Context will 
appear different than an identifier
parsed in another Context, even if they have the same textual name. Do not try to reuse parse trees or other objects 
created in one Context in another Context.
 
Each instance of the {\tt IAPI} interface creates its own Context object and most methods on 
that {\tt IAPI} instance operate with respect to that Context. The {\tt API.close} operation 
releases the Context object, allowing the garbage collector to reclaim space.
\footnote{The OpenJDK software was designed as a command-line tool, in which all memory is reclaimed
when the process exits. Although in principle memory can be garbage collected when no more references
to a Context or its consitutent parts exist, the degree to which this is the case has not been tested.}

\label{JavaFileObjects}
\subsection{JavaFileObjects}

OpenJDK works with source files using {\tt JavaFileObject} objects. This class abstracts the behavior of 
ordinary source files. Recall that the definition of the Java language allows source material to be held 
in containers other than ordinary files on disk; The {\tt JavaFileObject} class accommodates such implementations.

OpenJML currently handles source material in ordinary files and source material expressed as {\tt String} objects
and contained in mock-file objects. Such mock objects make it easier to create source material programatically, 
without having to create temporary files on disk.

Although the basic input unit to OpenJDK and OpenJML is a JavaFileObject, for convenience, methods that require
source material as input have variations allowing the inputs to be expressed as names of files or {\tt File} objects.
If needed, the following methods create JavaFileObjects:
\boxedexampleA{
String filename = ... \\
File file = new java.io.File(filename);\\
IAPI m = Factory.makeAPI();\\
JavaFileObject jfo1 = m.makeJFOfromFilename(filename);\\
JavaFileObject jfo2 = m.makeJFOfromFile(file);\\
JavaFileObject jfo3 = m.makeJFOfromString(filename,contents);
}
The last of the methods above, {\tt makeJFOfromString}, creates a mock-file object with the given contents (a String).
The {\tt contents} argument is a String holding the text that would be in a compilation unit.
The mock-object must have a sensible filename as well. In particular, the given filename should match the package and
class name as given in the {\tt contents} argument. In addition to creating the {\tt JavaFileObject} object, the 
mock-file is also added to an internal database of source mock-files; if a mock-file has a filename that would be on
the source path (were it a concrete file), then the mock-file is used as if it were a real file in an OpenJML compilation.
[TODO: Test this. Also, how to remove such files from the internal database. ] 

\subsection{Interfaces and concrete classes}

A design meant to be extended should preferably be expressed as Java interfaces; if client code uses the interface and not the underlying concrete classes, then reimplementing functionality with new classes is straightforward. The OpenJDK 
architecture uses interfaces in some places, but often it is the concrete classes that must be extended.

Table \ref{Interfaces} lists important interfaces, the corresponding OpenJDK concrete class, and the OpenJML replacement.

\begin{table}
\begin{tabular}{|l|l|l|}
\hline
Interface & OpenJDK class & OpenJML class \\ \hline
IAPI &  & API \\ \hline
 & com.sun.tools.javac.main.Main & org.jmlspecs.openjml.Main \\ \hline
 & Option & \\ \hline
IOption & & JmlOption \\ \hline
IVisitor & & \\ \hline
IJmlTree & & \\ \hline
IJmlVisitor & & \\ \hline
IProver & & \\ \hline
IProverResult & & ProverResult \\ \hline
IProverResult.ICounterexample & & Counterexample \\ \hline
IProverResult.ICoreIds & & \\ \hline
JCDiagnostic.DiagnosticPosition & SimpleDiagnosticPosition & DiagnosticPositionSE, DiagnosticPositionSES \\ \hline
Diagnostic<T> & JCDiagnostic & \\ \hline
 & com.sun.tools.javac.main.JavaCompiler & JmlCompiler \\ \hline
 & & \\ \hline
 & & \\ \hline
\end{tabular}
\caption{Interfaces and Classes}
\label{Interfaces}
\end{table}

TODO: Add Parser, Scanner, other tools, JCTree nodes, JMLTree nodes, Option/JmlOption, DiagnosticPosition, Tool, OptionCHecker


\subsection{Object Factories}

\subsection{Abstract Syntax Trees}

\subsection{Compilation Phases and The tool registry}

Compilation in the OpenJDK compiler proceeds in a number of phases. Each phase is implemented by a specific tool.
OpenJDK examples are the {\tt DocCommentScanner}, {\tt EndPosParser}, {\tt Flow}, performing scanning, parsing and flow checks respectively; the OpenJML counterparts are {\tt JmlScanner}, {\tt JmlParser}, and {\tt JmlFlow}.

In each compilation context there is one instance of each tool, registered with the context. The Context contains a map
of keys to the singleton instance of the tool (or its factory) for that context. The scanner and parser are treated slightly differently: there is a singleton instance of a scanner factory and a parser factory, but a new instance of the
scanner and the parser are created for each compilation unit compiled. Tables \ref{Tools1} and \ref{Tools2} list the tools
most likely to be encounterded when programming with OpenJML.

OpenJML implements
alternate versions of many of the OpenJDK tools. The OpenJML versions are derived from the OpenJDK versions and are 
registered in the context in place of the OpenJDK versions. In that way, anywhere in the software that a tool is
obtained (using the syntax {\tt ZZZ.instance(context)} for a tool {\tt ZZZ}), the appropriate version and instance
of the tool
is produced.

In some cases, a {\em tool factory} is registered instead of a tool instance. Then a tool instance is created on the
first request for an instance of the tool. The reason for this is the following. Most tools use other tools and, for
efficiency, request instances of those tools in their constructors. Circular dependencies can easily arise among these
tool dependencies. Using factories helps mitigate this, though the problem still does easily arise.



\begin{table}
\begin{tabularx}{\textwidth}{|p{1.5in}|p{1.5in}|X|}

\hline
Purpose & Java and JML tool &  Notes \\
\hline
overall compiler & JavaCompiler,\newline JmlCompiler & controls the flow of compilation phases \\
\hline
scanner factory & ScannerFactory,\newline JmlScanner.Factory  & \\
\hline
Token scanning & DocCommentScanner,\newline JmlScanner &  new instance created from the factory for each compilation unit \\
\hline
parser factory & ParserFactory,\newline JmlFactory  & \\
\hline
parser & EndPosParser,\newline JmlParser & new instance created from the factory for each compilation unit \\
\hline
symbol table construction & Enter,\newline JmlEnter & \\
\hline
annotation processing & Annotate & performed in JavaCompiler.processAnnotations\\
\hline
type determination and checking & Attr,\newline JmlAttr & \\
\hline
flow-sensitive checks & Flow,\newline JmlFlow & simple type-checking stops here \\
\hline
\hline
static checking & \hspace{.1in} \newline JmlEsc & invoked instead of desugaring if static checking is enabled (and processing ends here) \\
\hline
\hline
runtime assertion checking & \hspace*{.1in}\newline JmlRac & invoked if RAC is enabled, and then proceeds with the remainder of compilation and code generation \\
\hline
desugaring generics & & performed in the method JavaCompiler.desugar \\
\hline
code generation & Gen & not used for ESC \\
\hline


\end{tabularx}
\caption{Compilation phases and corresponding tools as implemented in JavaCompiler and JmlCompiler}
\label{Tools1}
\end{table}

\begin{table}[bth]
\begin{center}
\begin{tabular}{|l|l|l|}

\hline
Purpose & Java and JML tool &  Notes \\
\hline
identifier table & Names & \\
\hline
symbol table & SymTab & \\
\hline
compiler and command-line options & Options, JmlOptions & \\
\hline
AST node factory & JCTree.Factory, JmlTree.Maker & \\
\hline
message reporting & Log & \\
\hline
printing ASTs & Pretty, JmlPretty & \\
\hline
name resolution & Resolve, JmlResolver & \\
\hline
AST utilities & TreeInfo, JmlTreeInfo & \\
\hline
type checks & Check, JmlCheck & \\
\hline
creating diagnostic message objects & JCDiagnostic.Factory & \\
\hline
\end{tabular}
\end{center}
\caption{Some of the other registered tools}
\label{Tools2}
\end{table}

TBD: Others - MemberEnter, JmlMemberEnter, JmlRac, JmlCheck, Infer, Types, Options, Lint, Source, JavacMessages, DiagnosticListener, JavaFileManager/JavacFileManager, ClassReader/javadocClassReader, JavadocEnter, DocEnv/DocEnvJml, BasicBlocker, ProgressReporter?, ClassReader, ClassWriter, Todo, Annotate, Types, TaskListener, JavacTaskImpl, JavacTrees

TBD: Others - JmlSpecs, Utils, Nowarns, JmlTranslator, Dependencies

TBD: Is JmlTreeInfo still used

\section{OpenJML operations}

\subsection{Methods equivalent to command-line operations}

The {\tt execute} methods of {\tt IAPI} perform the same operation as a command on the command-line.
These methods are different than others of {\tt IAPI} in that they create and use their own {\tt Context}
object, ignoring that of the calling {\tt IAPI} object.

The simple method is shown here:
\boxedexampleA{
import org.jmlspecs.openjml.IAPI;\\
\\
IAPI m = new org.jmlspecs.openjml.API(); \\
int returnCode = m.execute(``-check'',''-noPurityCheck'',''src/demo/Err.java'');
}
\noindent
Each argument that would appear on the command-line is a separate argument to {\tt execute}.
All informational and diagnostic output is sent to {\tt System.out}.
The value returned by {\tt execute} is the same as the exit code returned by the equivalent command-line operation.
The String arguments are a varargs list, so they can be provided to {\tt execute} as a single array:
\boxedexampleA{
import org.jmlspecs.openjml.IAPI; \\
String[] args = new String[]\{``-check'',''-noPurityCheck'',''src/demo/Err.java''\}; \\
IAPI m = new org.jmlspecs.openjml.API(); \\
int returnCode = m.execute(args);
}
\noindent
A full example of using execute on a file with a syntax error is shown below:
\boxedinputc{\source/DemoExecute.java}{\source/DemoExecute.txt}

\noindent
A longer form of {\tt execute} takes two additional arguments: a {\tt Writer} and a
{\tt DiagnosticListener}. The {\tt Writer} receives all the informational output.
The {\tt report} method of the {\tt DiagnosticListener} is called for each warning or
error diagnostic generated by OpenJML. Here is a full example of this method:
\demo{DemoExecute2}


\subsection{Parsing}

There are two varieties of parsing. The first parses an individual Java or specification file, producing an AST that
represents that source file. The second parses both a Java file and its specification file, if there is a separate one.
The second form is generally more useful, since the specification file is found automatically. However, if the parse trees
are being constructed programmatically, it may be useful to parse the files individually and then manually associate them.

Parsing constructs a parse tree. No symbols are created or entered into a symbol table. Nor is any type-checking performed.
The only global effect is that identifiers are interned in the {\tt Names} table, which is specific to the compilation context.
Thus the only effect of discarding a parse tree is that there may be orphaned (no longer used) names in the {\tt Names} table.
The {\tt Names} table cannot be cleared without the risk of dangling identifiers in parse trees.

Other than this consideration, parse trees can be created, manipulated, edited and discarded. Section TBD describes tools for
manually creating parse trees and walking over them. Once a parse tree is type-checked, it should be considered immutable.

\subsubsection{Parsing individual files}

There are two methods for parsing an individual file. The basic 
method takes a {\tt JavaFileObject} as input and produces an AST.
The convenience method takes a filename as input and produces an AST. The methods of section \ref{JavaFileObjects} enable you to
produce {\tt JavaFileObject}s from filenames, File objects, or Strings that hold the equivalent of the contents of a file (a compilation unit).
\boxedexampleA{
JmlCompilationUnit parseSingleFile(String filename);\\
JmlCompilationUnit parseSingleFile(JavaFileObject jfo);
}
The filename is relative to the current working directory.
%% TBD: Change to reflect interfaces; check the JFO signature

Here is a full example that shows both interfaces and shows how to attach a specification parse tree to its Java parse tree.
\demo{DemoParseSingle}


%% TBD: How to attach specs to binary class files?

\subsubsection{Parsing Java and JML files together}

The more common action is to parse a Java file and its specification at the same time. The JML language defines how the specification file is found for a given source or binary class. In short, the specification file has syntax very similar to 
a Java file:
\begin{itemize}
\item it must be in the same package and have the same class name as the Java class
\item if both are files, the filenames without suffix must be the same
\item the specification file must be on the {\em specspath}
\item if a .jml file meeting the above criteria is found anywhere on the specspath, it is used; otherwise a .java file on the specspath meeting
the above criteria is used; otherwise only default specifications are used.\footnote{In the past, JML allowed multiple specification files and defined an ordering and rules for combining the specifications contained in them. The JML has been 
simplified to allow just one specification file, just one suffix (.jml), and no combining of specifications from a .jml and a .java file if both exist.}
\end{itemize}
Note that a Java file can be specified on the command-line that is not on the specspath. In that case (if there is no .jml file) 
no specification file will be found, although the user may expect that the Java file itself may serve as its own specifications.
This is a confusing situation and should be avoided.

%% TBD - more needed

\subsection{Type-checking}

\subsection{Static checking}

\subsection{Compiling run-time checks}

\subsection{Creating JML-enhanced documentation}

\section{Working with ASTs}

\subsection{Printing parse trees}

TBD

\subsection{Source location information}

TBD

\subsection{Exploring parse trees with Visitors}
OpenJML defines some Visitor classes that can be extended to implement user-defined functionality while traversing a parse tree.
The basic class is {\tt JmlScanner}. An unmodified instance of {\tt JmlScanner} will traverse a parse tree without performing
any actions.

There are three modes of traversing an AST.
\begin{itemize}
\item AST\_JAVA\_MODE - traverses only the Java portion of an AST, ignoring any JML annotations
\item AST\_JML\_MODE - traverses the Java and JML syntax that was part of the original source file
\item AST\_SPEC\_MODE - traverses the Java syntax and its specifications (whether they came from the same source file or a different one). This mode is only available after the AST has been type-checked.
\end{itemize}

A derived class can affect the behavior of the visitor in two ways:
\begin{itemize}
\item By overriding the {\tt scan} method, an action can be performed at every node of an AST
\item By overriding specific {\tt visit...} methods, an action can be performed that is specific to the nodes of the corresponding type
\end{itemize}

In the example that follows, the scan method of the Visitor is modified to print the node type and count all nodes in the AST, the visitBinary method
is modified to count Java binary operations, and the visitJmlBinary method is modified to count JML binary operations.
The default constructor of the parent Visitor class sets the traversal mode to AST\_JML\_MODE.
\demo{DemoWalkTree1}

The second example shows the differences among the three traversal modes. Note that the AST\_SPEC\_MODE traversal fails when requested prior to type-checking the AST.
\demo{DemoWalkTree2}

There are two other points to make about these examples.
\begin{itemize}
\item Note that each derived method calls the superclass version of the method that it overrides. The superclass method implements the logic to traverse all the children of the AST node. If the super call is omitted, no traversal of the children is performed. If the derived class wishes to traverse only some of the children, a specialized implementation of the method will need to be created.
It is easiest to create such an implementation by consulting the code in the super class.
\item In the examples above, you can see that the System.out.println statement that prints the node's class occurs before the super call. The result is a pre-order traversal of the tree; if the print statement occurred after the super call, the output would show a post-order traversal.
\end{itemize}

%% TBD - comment on other visitor classes

\subsection{Creating parse trees}

\section{Working with JML specifications}

\section{Utilities}

-- version
-- context
-- symbols

\section{Extending or modifying JML}
JML is modified by providing new implementations of key classes, typically by derivation from those that are part of OpenJML.
In fact, OpenJML extends many of the OpenJDK classes to incorporate JML functionality into the OpenJDK Java compiler.

\subsection{Adding new command-line options}
\subsection{Altering IAPI}
\subsection{Changing the Scanner}
\subsection{Enhancing the parser}
\subsection{Adding new modifiers and annotations}
\subsection{Adding new AST nodes}
\subsection{Modifying a compiler phase}




\part{Contributing to OpenJML}
\label{part:contributing}

\part{Semantics and translation of Java and JML in OpenJML}
\label{part:semantics}

\chapter{Introduction}

-- JML section is a summary of material from the reference manual

-- Sorted First-order-logic

-- individual subexpressions; optional expression form; optimization; usefulness for tracing

-- RAC vs. ESC

-- nomenclature



\chapter{Statement translations}

TODO: Need to insert both RAC and ESC in all of the following.

\section{While loop}

\noindent
Java and JML statement:

\noindent
{\tt }\\
{\tt //@ invariant} {\em\bf invariant\_condition} {\tt ;}\\
{\tt //@ decreases} {\em\bf counter} {\tt ;}\\
{\tt while (}{\em\bf condition) }{\tt \{}\\
\ind{\tt }{\em\bf body}\\
{\tt \}}\\

\noindent
Translation: {\em TODO: Needs variant condition, havoc information}

\noindent
{\tt }\\
{\tt \{}\\
\ind {\tt //@ assert }{\em jmltranslate({\bf invariant\_condition})} {\tt ;}\\
\ind {\tt //@ assert }{\em jmltranslate({\bf variant\_condition}) }{\tt > 0} {\tt ;}\\
\ind {\tt while (true) \{}\\
\ind\ind {\em stats(tmp,{\bf condition})}\\
\ind\ind {\tt if (!}{\em tmp}{) \{}\\
\ind\ind\ind {\tt //@ assume !}{\em tmp}{\tt ;}\\
\ind\ind\ind {\tt break;}\\
\ind\ind {\tt \}}\\
\ind\ind{\tt //@ assume }{\em tmp}{\tt ;}\\
\ind\ind {\tt }{\em stats({\bf body})}\\
\ind {\tt \} }\\
{\tt \}}

\chapter{Java expression translations}

\section{Implicit or explicit arithmetic conversions}

{\em TODO}

\section{Arithmetic expressions}
{\em TODO: need arithmetic range assertions}

In these, {\em T} is the type of the result of the operation. The two operands in binary operations are already assumed to have been converted to a common type according to Java's rules.

\noindent
{\em stats(tmp, }{\tt -} {\bf a} {\em )} ==> \\
\ind {\em stats(tmpa, }{\bf a} {\em )} \\
\ind {\em T tmp} {\tt =} {\tt - }{\em tmpa} {\tt ;} \\

\noindent
{\em stats(tmp, }{\bf a} {\tt +} {\bf b} {\em )} ==> \\
\ind {\em stats(tmpa, }{\bf a} {\em )} \\
\ind {\em stats(tmpb, }{\bf b} {\em )} \\
\ind {\em T tmp} {\tt =} {\em tmpa} {\tt +} {\em tmpb} {\tt ;} \\

\noindent
{\em stats(tmp, }{\bf a} {\tt -} {\bf b} {\em )} ==> \\
\ind {\em stats(tmpa, }{\bf a} {\em )} \\
\ind {\em stats(tmpb, }{\bf b} {\em )} \\
\ind {\em T tmp} {\tt =} {\em tmpa} {\tt -} {\em tmpb} {\tt ;}\\

\noindent
{\em stats(tmp, }{\bf a} {\tt *} {\bf b} {\em )} ==> \\
\ind {\em stats(tmpa, }{\bf a} {\em )} \\
\ind {\em stats(tmpb, }{\bf b} {\em )} \\
\ind {\em T tmp} {\tt =} {\em tmpa} {\tt *} {\em tmpb} {\tt ;}\\

\noindent
{\em stats(tmp, }{\bf a} {\tt /} {\bf b} {\em )} ==> \\
\ind {\em stats(tmpa, }{\bf a} {\em )} \\
\ind {\em stats(tmpb, }{\bf b} {\em )} \\
\ind {\tt //@ assert} {\em tmpb} {\tt != 0;} {\em // No division by zero}\\
\ind {\em T tmp} {\tt =} {\em tmpa} {\tt /} {\em tmpb} {\tt ;}\\

\noindent
{\em stats(tmp, }{\bf a} {\tt \%} {\bf b} {\em )} ==> \\
\ind {\em stats(tmpa, }{\bf a} {\em )} \\
\ind {\em stats(tmpb, }{\bf b} {\em )} \\
\ind {\tt //@ assert} {\em tmpb} {\tt != 0;} {\em // No division by zero}\\
\ind {\em T tmp} {\tt =} {\em tmpa} {\tt \%} {\em tmpb} {\tt ;}\\

\section{Bit-shift expressions}

{\em TODO}

\section{Relational expressions}

\noindent
No assertions are generated for the relational operations {\tt < > <= >= == !=}. The operands are presumed to have been converted to a common type according to Java's rules.\\

\noindent
{\em stats(tmp, }{\bf a} {\em op} {\bf b} {\em )} ==> \\
\ind {\em stats(tmpa, }{\bf a} {\em )} \\
\ind {\em stats(tmpb, }{\bf b} {\em )} \\
\ind {\em T tmp} {\tt =} {\em tmpa} {\em op} {\em tmpb} {\tt ;}\\

\section{Logical expressions}

\noindent
{\em stats(tmp,} {\tt !} {\bf a} {\em )} ==> \\
\ind {\em stats(tmpa, }{\bf a} {\em )} \\
\ind {\em T tmp} {\tt =} {\tt !} {\em tmpa} {\tt ;}\\

\noindent
The {\tt \&\&} and {\tt ||} operations are short-circuit operations in which the second operand is conditionally evaluated.
Here {\tt \&} and {\tt |} are the (FOL) boolean non-short-circuit conjunction and disjunction.\\

\noindent
{\em stats(tmp, }{\bf a} {\tt \&\&} {\bf b} {\em )} ==> \\
\ind {\tt boolean } {\em tmp} {\tt ;} \\
\ind {\em stats(tmpa, }{\bf a} {\em )} \\
\ind {\tt if (} {\em tmpa} {\tt ) \{} \\
\ind\ind {\tt //@ assume} {\em tmpa} {\tt ;} \\
\ind\ind {\em stats(tmpb, }{\bf b} {\em )} \\
\ind\ind {\em tmp} {\tt =} {\em tmpa} {\tt \&} {\em tmpb} {\tt ;}\\
\ind {\tt \} else \{}\\
\ind\ind {\tt //@ assume !} {\em tmpa} {\tt ;} \\
\ind\ind {\em tmp} {\tt =} {\em tmpa} {\tt ;}\\
\ind {\tt \}}\\

\noindent
{\em stats(tmp, }{\bf a} {\tt ||} {\bf b} {\em )} ==> \\
\ind {\tt boolean } {\em tmp} {\tt ;} \\
\ind {\em stats(tmpa, }{\bf a} {\em )} \\
\ind {\tt if ( !} {\em tmpa} {\tt ) \{} \\
\ind\ind {\tt //@ assume !} {\em tmpa} {\tt ;} \\
\ind\ind {\em stats(tmpb, }{\bf b} {\em )} \\
\ind\ind {\em tmp} {\tt =} {\em tmpa} {\tt |} {\em tmpb} {\tt ;}\\
\ind {\tt \} else \{}\\
\ind\ind {\tt //@ assume} {\em tmpa} {\tt ;} \\
\ind\ind {\em tmp} {\tt =} {\em tmpa} {\tt ;}\\
\ind {\tt \}}\\


\pagebreak

TODOs
\begin{itemize}
\item Fix the TITLE for the web pages
\item on HTML pages boxed examples do not render correctly
\end{itemize}

\textit{An index will be added later.} %% TBD


%% TBD - index
%% TBD - bibliography

\end{document}
